% !TeX spellcheck = en-US
% !TeX encoding = utf8
% !TeX program = pdflatex
% !BIB program = biber
% -*- coding:utf-8 mod:LaTeX -*-


% vv  scroll down to line 200 for content  vv


\let\ifdeutsch\iffalse
\let\ifenglisch\iftrue
% EN: This file is loaded before the \documentclass command in the main document

% EN: The following package allows \\ at the title page
%     For more information see https://github.com/latextemplates/scientific-thesis-cover/issues/4
\RequirePackage{kvoptions-patch}

\ifenglisch
  \PassOptionsToClass{numbers=noenddot}{scrbook}
\else
  %()Aus scrguide.pdf - der Dokumentation von KOMA-Script)
  %Nach DUDEN steht in Gliederungen, in denen ausschließlich arabische Ziffern für die Nummerierung
  %verwendet werden, am Ende der Gliederungsnummern kein abschließender Punkt
  %(siehe [DUD96, R3]). Wird hingegen innerhalb der Gliederung auch mit römischen Zahlen
  %oder Groß- oder Kleinbuchstaben gearbeitet, so steht am Ende aller Gliederungsnummern ein
  %abschließender Punkt (siehe [DUD96, R4])
  \PassOptionsToClass{numbers=autoendperiod}{scrbook}
\fi

% Warns about outdated packages and missing caption declarations
% See https://www.ctan.org/pkg/nag
\RequirePackage[l2tabu, orthodox]{nag}

%DE: Neue deutsche Trennmuster
%    Siehe http://www.ctan.org/pkg/dehyph-exptl und http://projekte.dante.de/Trennmuster/WebHome
%    Nur für pdflatex, nicht für lualatex
\RequirePackage{ifluatex}
\ifluatex
  % do not load anything
\else
  \ifdeutsch
    \RequirePackage[ngerman=ngerman-x-latest]{hyphsubst}
  \fi
\fi

\documentclass[
  % fontsize=11pt is the standard
  a4paper,  % Standard format - only KOMAScript uses paper=a4 - https://tex.stackexchange.com/a/61044/9075
  twoside,  % we are optimizing for both screen and two-side printing. So the page numbers will jump, but the content is configured to stay in the middle (by using the geometry package)
  bibliography=totoc,
  %               idxtotoc,   %Index ins Inhaltsverzeichnis
  %               liststotoc, %List of X ins Inhaltsverzeichnis, mit liststotocnumbered werden die Abbildungsverzeichnisse nummeriert
  headsepline,
  cleardoublepage=empty,
  parskip=half,
  %               draft    % um zu sehen, wo noch nachgebessert werden muss - wichtig, da Bindungskorrektur mit drin
  draft=false
]{scrbook}
% !TeX encoding = utf8
% -*- coding:utf-8 mod:LaTeX -*-

% EN: This file includes basic packages and sets options. The order of package
%     loading is important

% DE: In dieser Datei werden zuerst die benoetigten Pakete eingebunden und
%     danach diverse Optionen gesetzt. Achtung Reihenfolge ist entscheidend!


% EN: Styleguide:
% - English comments are prefixed with "EN", German comments are prefixed with "DE"
% - Prefixed headings define the language for the subsequent paragraphs
% - It is tried to organize packages in blocks. Bocks are separated by two empty lines.

% DE: Styleguide:
%
% Ein sehr kleiner Styleguide. Packages werden in Blöcken organisiert.
% Zwischen zwei Blöcken sind 2 Leerzeilen!


% EN: Enable copy and paste of text from the PDF
%     Only required for pdflatex. It "just works" in the case of lualatex.
%     mmap enables mathematical symbols, but does not work with the newtx font set
%     See: https://tex.stackexchange.com/a/64457/9075
%     Other solutions outlined at http://goemonx.blogspot.de/2012/01/pdflatex-ligaturen-und-copynpaste.html and http://tex.stackexchange.com/questions/4397/make-ligatures-in-linux-libertine-copyable-and-searchable
%     Trouble shooting outlined at https://tex.stackexchange.com/a/100618/9075

\ifluatex
\else
  \usepackage{cmap}
\fi


% EN: File encoding
% DE: Codierung
%     Wir sind im 21 Jahrhundert, utf-8 löst so viele Probleme.
%
% Mit UTF-8 funktionieren folgende Pakete nicht mehr. Bitte beachten!
%   * fancyvrb mit §
%   * easylist -> http://www.ctan.org/tex-archive/macros/latex/contrib/easylist/
\ifluatex
  % EN: See https://tex.stackexchange.com/a/158517/9075
  %     Not required, because of usage of fontspec package
  %\usepackage[utf8]{luainputenc}
\else
  \usepackage[utf8]{inputenc}
\fi


% DE: Parallelbetrieb tex4ht und pdflatex

\makeatletter
\@ifpackageloaded{tex4ht}{
  \def\iftex4ht{\iftrue}
}{
  \def\iftex4ht{\iffalse}
}
\makeatother


% EN: Mathematics
% DE: Mathematik
%
% DE: Viele Mathematik-Sachen. Siehe https://texdoc.net/pkg/amsmath
%
% EN: Options must be passed this way, otherwise it does not work with glossaries
% DE: fleqn (=Gleichungen linksbündig platzieren) funktioniert nicht direkt. Es muss noch ein Patch gemacht werden:
\PassOptionsToPackage{fleqn,leqno}{amsmath}
%
% DE: amsmath Muss nicht mehr geladen werden, da es von newtxmath automatisch geladen wird
% \usepackage{amsmath}


%% EN: Fonts
%% DE: Schriften
%%
%% !!! If you change the font, be sure that words such as "workflow" can
%% !!! still be copied from the PDF. If this is not the case, you have
%% !!! to use glyphtounicode. See comment at cmap package


% EN: Times Roman for all text
\ifluatex
  % source: Second proposed fix from the following answer: https://tex.stackexchange.com/a/394137
  \usepackage[no-math]{fontspec}
  \setmainfont{TeXGyreTermes-Regular}[
       BoldFont       = TeXGyreTermes-Bold ,
       ItalicFont     = TeXGyreTermes-Italic ,
       BoldItalicFont = TeXGyreTermes-BoldItalic,
       NFSSFamily     = ntxtlf]
  \setsansfont{TeX Gyre Heros Regular}[
       Scale=.9,
       BoldFont       = TeX Gyre Heros Bold,
       ItalicFont     = TeX Gyre Heros Italic,
       BoldItalicFont = TeX Gyre Heros BoldItalic]
  \setmonofont[StylisticSet={1,3},Scale=.9]{inconsolata}
  \RequirePackage{newtxmath}
\else
  \RequirePackage{newtxtext}
  \RequirePackage{newtxmath}
  % EN: looks good with times, but no equivalent for lualatex found,
  %     therefore replaced with inconsolata
  %\RequirePackage[zerostyle=b,scaled=.9]{newtxtt}
  \RequirePackage[varl,scaled=.9]{inconsolata}
\fi

% EN: Fallback font - if the subsequent font packages do not define a font (e.g., monospaced)
%     This is the modern package for "Computer Modern".
%     In case this gets activated, one has to switch from cmap package to glyphtounicode (in the case of pdflatex)
% DE: Fallback-Schriftart
%\usepackage[%
%    rm={oldstyle=false,proportional=true},%
%    sf={oldstyle=false,proportional=true},%
%    tt={oldstyle=false,proportional=true,variable=true},%
%    qt=false%
%]{cfr-lm}

% EN: Headings are typset in Helvetica (which is similar to Arial)
% DE: Schriftart fuer die Ueberschriften - ueberschreibt lmodern
%\usepackage[scaled=.95]{helvet}

% DE: Für Schreibschrift würde tun, muss aber nicht
%\usepackage{mathrsfs} %  \mathscr{ABC}

% EN: Font for the main text
% DE: Schriftart fuer den Fliesstext - ueberschreibt lmodern
%     Linux Libertine, siehe http://www.linuxlibertine.org/
%     Packageparamter [osf] = Minuskel-Ziffern
%     rm = libertine im Brottext, Linux Biolinum NICHT als serifenlose Schrift, sondern helvet (von oben) beibehalten
%\usepackage[rm]{libertine}

% EN: Alternative Font: Palantino. It is recommeded by Prof. Ludewig for German texts
% DE: Alternative Schriftart: Palantino, Packageparamter [osf] = Minuskel-Ziffern
%     Bitte nur in deutschen Texten
%\usepackage{mathpazo} %ftp://ftp.dante.de/tex-archive/fonts/mathpazo/ - Tipp aus DE-TEX-FAQ 8.2.1

% DE: Schriftart fuer Programmcode - ueberschreibt lmodern
%     Falls auskommentiert, wird die Standardschriftart lmodern genommen
%     Fuer schreibmaschinenartige Schluesselwoerter in den Listings - geht bei alten Installationen nicht, da einige Fontshapes (<>=) fehlen
%\usepackage[scaled=.92]{luximono}
%\usepackage{courier}
% DE: BeraMono als Typewriter-Schrift, Tipp von http://tex.stackexchange.com/a/71346/9075
%\usepackage[scaled=0.83]{beramono}

% EN: backticks (`) are rendered as such in verbatim environments.
%     See following links for details:
%     - https://tex.stackexchange.com/a/341057/9075
%     - https://tex.stackexchange.com/a/47451/9075
%     - https://tex.stackexchange.com/a/166791/9075
\usepackage{upquote}

% DE: Symbole
%
%\usepackage[geometry]{ifsym} % \BigSquare
%\usepackage{mathabx}
%\usepackage{stmaryrd} %fuer \ovee, \owedge, \otimes
%\usepackage{marvosym} %fuer \Writinghand %patched to not redefine \Rightarrow
%\usepackage{mathrsfs} %mittels \mathscr{} schoenen geschwungenen Buchstaben erzeugen
%\usepackage{calrsfs} %\mathcal{} ein bisserl dickeren buchstaben erzeugen - sieht net so gut aus.
%durch mathpazo ist das schon definiert

%
%\usepackage{amssymb}

% EN: For \texttrademark{}
\usepackage{textcomp}

% EN: name-clashes von marvosym und mathabx vermeiden:
\def\delsym#1{%
  %  \expandafter\let\expandafter\origsym\expandafter=\csname#1\endcsname
  %  \expandafter\let\csname orig#1\endcsname=\origsym
  \expandafter\let\csname#1\endcsname=\relax
}

%\usepackage{pifont}
%\usepackage{bbding}
%\delsym{Asterisk}
%\delsym{Sun}\delsym{Mercury}\delsym{Venus}\delsym{Earth}\delsym{Mars}
%\delsym{Jupiter}\delsym{Saturn}\delsym{Uranus}\delsym{Neptune}
%\delsym{Pluto}\delsym{Aries}\delsym{Taurus}\delsym{Gemini}
%\delsym{Rightarrow}
%\usepackage{mathabx} - Ueberschreibt leider zu viel - und die \le-Zeichen usw. sehen nicht gut aus!


% EN: Modern font encoding
%     Has to be loaded AFTER any font packages. See https://tex.stackexchange.com/a/2869/9075.
\ifluatex
\else
  \usepackage[T1]{fontenc}
\fi
%


% EN: Character protrusion and font expansion. See http://www.ctan.org/tex-archive/macros/latex/contrib/microtype/
% DE: Optischer Randausgleich und Grauwertkorrektur

\usepackage[
  babel=true, % EN: Enable language-specific kerning. Take language-settings from the languge of the current document (see Section 6 of microtype.pdf)
  expansion=alltext,
  protrusion=alltext-nott, % EN: Ensure that at listings, there is no change at the margin of the listing
  final % EN: Always enable microtype, even if in draft mode. This helps finding bad boxes quickly.
        %     In the standard configuration, this template is always in the final mode, so this option only makes a difference if "pros" use the draft mode
]{microtype}


% EN: \texttt{test -- test} keeps the "--" as "--" (and does not convert it to an en dash)
\DisableLigatures{encoding = T1, family = tt* }

% DE: fuer microtype
% DE: tracking=true muss als Parameter des microtype-packages mitgegeben werden
% DE: Deaktiviert, da dies bei Algorithmen seltsam aussieht

%\DeclareMicrotypeSet*[tracking]{my}{ font = */*/*/sc/* }%
%\SetTracking{ encoding = *, shape = sc }{ 45 }
% DE: Hier wird festgelegt,
%     dass alle Passagen in Kapitälchen automatisch leicht
%     gesperrt werden.
%     Quelle: http://homepage.ruhr-uni-bochum.de/Georg.Verweyen/pakete.html
%    Deaktiviert, da sonst "BPEL", "BPMN" usw. wirklich komisch aussehen.
%     Macht wohl nur bei geisteswissenschaftlichen Arbeiten Sinn.


% EN: amsmath teaks


% EN: Fixes bugs in AMS math
%     Corrently conflicts with unicode-math
% \usepackage{mathtools}

%\numberwithin{equation}{section}
%\renewcommand{\theequation}{\thesection.\Roman{equation}}

% EN: work-around ams-math problem with align and 9 -> 10. Does not work with glossaries, No visual changes.
%\addtolength\mathindent{1em}


% EN: For theorems, replacement for amsthm
\usepackage[amsmath,hyperref]{ntheorem}
\theorempreskipamount 2ex plus1ex minus0.5ex
\theorempostskipamount 2ex plus1ex minus0.5ex
\theoremstyle{break}
\newtheorem{definition}{Definition}[section]


% CTAN: https://ctan.org/pkg/lccaps
% Doc: http://texdoc.net/pkg/lccaps
%
% Required for DE/EN \initialism
\usepackage{lccaps}


% EN: Defintion of colors. Argument "hyperref" is not used as we do not want to change border colors of links: Links are not colored anymore.
% DE: Farbdefinitionen
\usepackage[dvipsnames]{xcolor}


% EN: Required for custom acronyms/glossaries style.
%     Left aligned Columns in tables with fixed width.
%     See http://tex.stackexchange.com/questions/91566/syntax-similar-to-centering-for-right-and-left
\usepackage{ragged2e}


% DE: Wichtig, ansonsten erscheint "No room for a new \write"
\usepackage{scrwfile}


% EN: Support for language-specific hyphenation
% DE: Neue deutsche Rechtschreibung und Literatur statt "Literature"
%     Die folgende Einstellung ist der Nachfolger von ngerman.sty
\ifdeutsch
  % DE: letzte Sprache ist default, Einbindung von "american" ermöglicht \begin{otherlanguage}{amercian}...\end{otherlanguage} oder \foreignlanguage{american}{Text in American}
  %     Siehe auch http://tex.stackexchange.com/a/50638/9075
  \usepackage[american,main=ngerman]{babel}
  % Ein "abstract" ist eine "Kurzfassung", keine "Zusammenfassung"
  \addto\captionsngerman{%
    \renewcommand\abstractname{Kurzfassung}%
  }
  \ifluatex
    % EN: conditionally disable ligatures. See https://github.com/latextemplates/scientific-thesis-template/issues/54
    %     for a discussion
    \usepackage[ngerman]{selnolig}
  \fi
\else
  % EN: Set English as language and allow to write hyphenated"=words
  %     `american`, `english` and `USenglish` are synonyms for babel package (according to https://tex.stackexchange.com/questions/12775/babel-english-american-usenglish).
  %      "english" has to go last to set it as default language
  \usepackage[ngerman,main=english]{babel}
  % EN: Hint by http://tex.stackexchange.com/a/321066/9075 -> enable "= as dashes
  \addto\extrasenglish{\languageshorthands{ngerman}\useshorthands{"}}
  \ifluatex
    % EN: conditionally disable ligatures. See https://github.com/latextemplates/scientific-thesis-template/issues/54
    %     for a discussion
    \usepackage[english]{selnolig}
  \fi
\fi
%


% EN: For easy quotations: \enquote{text}
%     This package is very smart when nesting is applied, otherwise textcmds (see below) provides a shorter command
%     Note that this package results in a warning when it is loaded before minted (actually fvextra).
% DE: Anführungszeichen
%     Zitate in \enquote{...} setzen, dann werden automatisch die richtigen Anführungszeichen verwendet.
%     Dieses package erzeugt eine Warnung, wenn es vor minted (genauer fvextra) geladen wird.
\usepackage{csquotes}


% EN: For even easier quotations: \qq{text}.
%     Is not smart in the case of nesting, but good enough for the most cases
\usepackage{textcmds}
\ifdeutsch
  % EN: German quotes are different. So do not use the English quotes, but the ones provided by the csquotes package.
  \renewcommand{\qq}[1]{\enquote{#1}}
\fi


% EN: extended enumarations
% DE: erweitertes Enumerate
\usepackage{paralist}


% DE: Gestaltung der Kopf- und Fußteilen

\usepackage[automark]{scrlayer-scrpage}

\automark[section]{chapter}
\setkomafont{pageheadfoot}{\normalfont\sffamily}
\setkomafont{pagenumber}{\normalfont\sffamily}

% DE: funktioniert nicht: Alle Linien sind hier weg
%\setheadsepline[.4pt]{.4pt}


% DE: Intelligentes Leerzeichen um hinter Abkürzungen die richtigen Abstände zu erhalten, auch leere.
%     Siehe commands.tex \gq{}
\usepackage{xspace}
% DE: Macht \xspace und \enquote kompatibel
\makeatletter
\xspaceaddexceptions{\grqq \grq \csq@qclose@i \} }
\makeatother


\newcommand{\eg}{e.\,g.,\ }
\newcommand{\ie}{i.\,e.,\ }


% EN: introduce \powerset - hint by http://matheplanet.com/matheplanet/nuke/html/viewtopic.php?topic=136492&post_id=997377
\DeclareFontFamily{U}{MnSymbolC}{}
\DeclareSymbolFont{MnSyC}{U}{MnSymbolC}{m}{n}
\DeclareFontShape{U}{MnSymbolC}{m}{n}{
  <-6>    MnSymbolC5
  <6-7>   MnSymbolC6
  <7-8>   MnSymbolC7
  <8-9>   MnSymbolC8
  <9-10>  MnSymbolC9
  <10-12> MnSymbolC10
  <12->   MnSymbolC12%
}{}
\DeclareMathSymbol{\powerset}{\mathord}{MnSyC}{180}


% EN: Package for the appendix
% DE: Anhang
\usepackage{appendix}
%[toc,page,title,header]
%


% EN: Graphics
% DE: Grafikeinbindungen
%
% EN: The parameter "pdftex" is not required
\usepackage{graphicx}
\graphicspath{{\getgraphicspath}}
\newcommand{\getgraphicspath}{graphics/}


% EN: Enables inclusion of SVG graphics - 1:1 approach
%    This is NOT the approach of https://ctan.org/pkg/svg-inkscape,
%     which allows text in SVG to be typeset using LaTeX
%     We just include the SVG as is.
\usepackage{epstopdf}
\epstopdfDeclareGraphicsRule{.svg}{pdf}{.pdf}{%
  inkscape -z -D --file=#1 --export-pdf=\OutputFile
}


% EN: Enables inclusion of SVG graphics - text-rendered-with-LaTeX-approach
%     This is the approach of https://ctan.org/pkg/svg-inkscape,
\newcommand{\executeiffilenewer}[3]{%
  \IfFileExists{#2}
  {
    %\message{file #2 exists}
    \ifnum\pdfstrcmp{\pdffilemoddate{#1}}%
      {\pdffilemoddate{#2}}>0%
      {\immediate\write18{#3}}
    \else
      {%\message{file up to date #2}
      }
    \fi%
  }{
    %\message{file #2 doesn't exist}
    %\message{argument: #3}
    %\immediate\write18{echo "test" > xoutput.txt}
    \immediate\write18{#3}
  }
}
\newcommand{\includesvg}[1]{%
  \executeiffilenewer{#1.svg}{#1.pdf}%
  {
    inkscape -z -D --file=\getgraphicspath#1.svg %
    --export-pdf=\getgraphicspath#1.pdf --export-latex}%
  \input{\getgraphicspath#1.pdf_tex}%
}


% EN: Enable typesetting values with SI units.
\ifdeutsch
  \usepackage[mode=text,group-four-digits]{siunitx}
  \sisetup{locale=DE}
\else
  \usepackage[mode=text,group-four-digits,group-separator={,}]{siunitx}
  \sisetup{locale=US}
\fi


% EN: Extensions for tables
% DE: Tabellenerweiterungen
\usepackage{array} %increases tex's buffer size and enables ``>'' in tablespecs
\usepackage{longtable}
\usepackage{dcolumn} %Aligning numbers by decimal points in table columns
\ifdeutsch
  \newcolumntype{d}[1]{D{.}{,}{#1}}
\else
  \newcolumntype{d}[1]{D{.}{.}{#1}}
\fi
\setlength{\extrarowheight}{1pt}


% DE: Eine Zelle, die sich über mehrere Zeilen erstreckt.
%     Siehe Beispieltabelle in Kapitel 2
\usepackage{multirow}


% DE: Fuer Tabellen mit Variablen Spaltenbreiten
%\usepackage{tabularx}
%\usepackage{tabulary}


% EN: Links behave as they should. Enables "\url{...}" for URL typesettings.
%     Allow URL breaks also at a hyphen, even though it might be confusing: Is the "-" part of the address or just a hyphen?
%     See https://tex.stackexchange.com/a/3034/9075.
% DE: Links verhalten sich so, wie sie sollen
%     Zeilenumbrüche bei URLs auch bei Bindestrichen erlauben, auch wenn es verwirrend sein könnte: Gehört der Bindestrich zur URL oder ist es ein Trennstrich?
%     Siehe https://tex.stackexchange.com/a/3034/9075.
\usepackage[hyphens]{url}
%
%  EN: When activated, use text font as url font, not the monospaced one.
%      For all options see https://tex.stackexchange.com/a/261435/9075.
% \urlstyle{same}
%
% EN: Hint by http://tex.stackexchange.com/a/10419/9075.
\makeatletter
\g@addto@macro{\UrlBreaks}{\UrlOrds}
\makeatother


% DE: Index über Begriffe, Abkürzungen
%\usepackage{makeidx} makeidx ist out -> http://xindy.sf.net verwenden


% DE: lustiger Hack fuer das Abkuerzungsverzeichnis
%     nach latex durchlauf folgendes ausfuehren
%     makeindex ausarbeitung.nlo -s nomencl.ist -o ausarbeitung.nls
%     danach nochmal latex
%\usepackage{nomencl}
%    \let\abk\nomenclature %Deutsche Ueberschrift setzen
%          \renewcommand{\nomname}{List of Abbreviations}
%        %Punkte zw. Abkuerzung und Erklaerung
%          \setlength{\nomlabelwidth}{.2\hsize}
%          \renewcommand{\nomlabel}[1]{#1 \dotfill}
%        %Zeilenabstaende verkleinern
%          \setlength{\nomitemsep}{-\parsep}
%    \makenomenclature


% EN: Logic for TeX - enables if-then-else in commands
% DE: Logik für TeX
%     FÜr if-then-else @ commands.tex
\usepackage{ifthen}

\usepackage[algo2e, ruled, vlined, linesnumbered]{algorithm2e}

% EN: Code Listings
% DE: Listings
\usepackage{listings}
\lstset{language=XML,
  showstringspaces=false,
  extendedchars=true,
  basicstyle=\footnotesize\ttfamily,
  commentstyle=\slshape,
  % DE: Original: \rmfamily, damit werden die Strings im Quellcode hervorgehoben. Zusaetzlich evtl.: \scshape oder \rmfamily durch \ttfamily ersetzen. Dann sieht's aus, wie bei fancyvrb
  stringstyle=\ttfamily,
  breaklines=true,
  breakatwhitespace=true,
  % EN: alternative: fixed
  columns=flexible,
  numbers=left,
  numberstyle=\tiny,
  basewidth=.5em,
  xleftmargin=.5cm,
  % aboveskip=0mm, %DE: deaktivieren, falls man lstlistings direkt als floating object benutzt (\begin{lstlisting}[float,...])
  % belowskip=0mm, %DE: deaktivieren, falls man lstlistings direkt als floating object benutzt (\begin{lstlisting}[float,...])
  captionpos=b
}

\ifluatex
\else
  % EN: Enable UTF-8 support - see https://tex.stackexchange.com/q/419327/9075
  \usepackage{listingsutf8}
  \lstset{inputencoding=utf8/latin1}
\fi

\ifdeutsch
  \renewcommand{\lstlistlistingname}{Verzeichnis der Listings}
\fi


% EN: Alternative to listings could be fancyvrb. Can be used together.
% DE: Alternative zu Listings ist fancyvrb. Kann auch beides gleichzeitig benutzt werden.
\usepackage{fancyvrb}
%
% EN: Font size for the normal text
% DE: Groesse fuer den Fliesstext. Falls deaktiviert: \normalsize
%\fvset{fontsize=\small}
%
% DE: Somit kann im Text ganz einfach §verbatim§ text gesetzt werden.
%     Disabled, because UTF-8 does not work any more and lualatex causes issues
%\DefineShortVerb{\§}
%
% EN: Shrink font size of listings
\RecustomVerbatimEnvironment{Verbatim}{Verbatim}{fontsize=\footnotesize}
\RecustomVerbatimCommand{\VerbatimInput}{VerbatimInput}{fontsize=\footnotesize}
%
% EN: Hack for fancyvrb based on http://newsgroups.derkeiler.com/Archive/Comp/comp.text.tex/2008-12/msg00075.html
%     Change of the solution: \Vref somehow collidated with cleveref/varioref as the output of \Vref{} was "Abschnitt 4.3 auf Seite 85"; therefore changed to \myVref -- so completely removed
%     See https://tex.stackexchange.com/q/132420/9075 for more information.
\newcommand{\Vlabel}[1]{\label[line]{#1}\hypertarget{#1}{}}
\newcommand{\lref}[1]{\hyperlink{#1}{\FancyVerbLineautorefname~\ref*{#1}}}


% EN: Tunings of captions for floats, listings, ...
% DE: Bildunterschriften bei floats genauso formatieren wie bei Listings
%     Anpassung wird unten bei den newfloat-Deklarationen vorgenommen
%     https://www.ctan.org/pkg/caption2 is superseeded by this package.
\usepackage{caption}


% EN: Provides rotating figures, where the PDF page is also turned
% DE: Ermoeglicht es, Abbildungen um 90 Grad zu drehen
%     Alternatives Paket: rotating Allerdings wird hier nur das Bild gedreht, während bei lscape auch die PDF-Seite gedreht wird.
%     Das Paket lscape dreht die Seite auch nicht
\usepackage{pdflscape}


% EN: Required for proper environments of fancyvrb and lstlistings
%    There is also the newfloat pacakge (recommended by minted), but we currently have no expericene with that
% DE: Wird für fancyvrb und für lstlistings verwendet
\usepackage{float}
%
% EN: Alternative to float package
%\usepackage{floatrow}
% DE: zustäzlich für den Paramter [H] = Floats WIRKLICH da wo sie deklariert wurden paltzieren - ganz ohne Kompromisse
%     floatrow ist der Nachfolger von float
%     Allerdings macht floatrow in manchen Konstellationen Probleme. Deshalb ist das Paket deaktiviert.
%
% EN: See http://www.tex.ac.uk/cgi-bin/texfaq2html?label=floats
% DE: floats IMMER nach einer Referenzierung platzieren
%\usepackage{flafter}


% EN: Put footnotes below floats
%     Source: https://tex.stackexchange.com/a/32993/9075
\usepackage{stfloats}
\fnbelowfloat


% EN: For nested figures
% DE: Fuer Abbildungen innerhalb von Abbildungen
%     Ersetzt die Pakete subfigure und subfig - siehe https://tex.stackexchange.com/a/13778/9075
\usepackage[hypcap=true]{subcaption}


% EN: Extended support for footnotes
% DE: Fußnoten
%
%\usepackage{dblfnote}  %Zweispaltige Fußnoten
%
% Keine hochgestellten Ziffern in der Fußnote (KOMA-Script-spezifisch):
%\deffootnote[1.5em]{0pt}{1em}{\makebox[1.5em][l]{\bfseries\thefootnotemark}}
%
% Abstand zwischen Fußnoten vergrößern:
%\setlength{\footnotesep}{.85\baselineskip}
%
% EN: Following command disables the separting line of the footnote
% DE: Folgendes Kommando deaktiviert die Trennlinie zur Fußnote
%\renewcommand{\footnoterule}{}
%
\addtolength{\skip\footins}{\baselineskip} % Abstand Text <-> Fußnote
%
% Fußnoten immer ganz unten auf einer \raggedbottom-Seite
% fnpos kommt aus dem yafoot package
\usepackage{fnpos}
\makeFNbelow
\makeFNbottom


% EN: Variable page heights
% DE: Variable Seitenhöhen zulassen
\raggedbottom


% DE: Falls die Seitenzahl bei einer Referenz auf eine Abbildung nur dann angegeben werden soll,
%     falls sich die Abbildung nicht auf der selben Seite befindet...
\iftex4ht
  %tex4ht does not work well with vref, therefore we emulate vref behavior
  \newcommand{\vref}[1]{\ref{#1}}
\else
  \ifdeutsch
    \usepackage[ngerman]{varioref}
  \else
    \usepackage{varioref}
  \fi
\fi


% EN: More beautiful tables if one uses \toprule, \midrule, \bottomrule
% DE: Noch schoenere Tabellen als mit booktabs mit http://www.zvisionwelt.de/downloads.html
\usepackage{booktabs}
%
%\usepackage[section]{placeins}


% EN: Graphs and Automata
%
% TODO: Since version 3.0 (2013-10-01), it supports pdflatex via the auto-pst-pdf package
%       Requires -shell-escape
%\usepackage{gastex}


%\usepackage{multicol}

% DE: kollidiert mit diplomarbeit.sty
%\usepackage{setspace}


% DE: biblatex statt bibtex
\usepackage[
  backend       = biber, %biber does not work with 64x versions alternative: bibtex8
  %minalphanames only works with biber backend
  sortcites     = true,
  bibstyle      = alphabetic,
  citestyle     = alphabetic,
  giveninits    = true,
  useprefix     = false, %"von, van, etc." will be printed, too. See below.
  minnames      = 1,
  minalphanames = 3,
  maxalphanames = 4,
  maxbibnames   = 99,
  maxcitenames  = 2,
  natbib        = true,
  eprint        = true,
  url           = true,
  doi           = true,
  isbn          = true,
  backref       = true]{biblatex}

% enable more breaks at URLs. See https://tex.stackexchange.com/a/134281.
\setcounter{biburllcpenalty}{7000}
\setcounter{biburlucpenalty}{8000}

\bibliography{bibliography}
%\addbibresource[datatype=bibtex]{bibliography.bib}

%Do not put "vd" in the label, but put it at "\citeauthor"
%Source: http://tex.stackexchange.com/a/30277/9075
\makeatletter
\AtBeginDocument{\toggletrue{blx@useprefix}}
\AtBeginBibliography{\togglefalse{blx@useprefix}}
\makeatother

%Thin spaces between initials
%http://tex.stackexchange.com/a/11083/9075
\renewrobustcmd*{\bibinitdelim}{\,}

%Keep first and last name together in the bibliography
%http://tex.stackexchange.com/a/196192/9075
\renewcommand*\bibnamedelimc{\addnbspace}
\renewcommand*\bibnamedelimd{\addnbspace}

%Replace last "and" by comma in bibliography
%See http://tex.stackexchange.com/a/41532/9075
\AtBeginBibliography{%
  \renewcommand*{\finalnamedelim}{\addcomma\space}%
}

\DefineBibliographyStrings{ngerman}{
  backrefpage  = {zitiert auf S\adddot},
  backrefpages = {zitiert auf S\adddot},
  andothers    = {et\ \addabbrvspace al\adddot},
  %Tipp von http://www.mrunix.de/forums/showthread.php?64665-biblatex-Kann-%DCberschrift-vom-Inhaltsverzeichnis-nicht-%E4ndern&p=293656&viewfull=1#post293656
  bibliography = {Literaturverzeichnis}
}

% EN: enable hyperlinked author names when using \citeauthor
%     source: http://tex.stackexchange.com/a/75916/9075
\DeclareCiteCommand{\citeauthor}
{\boolfalse{citetracker}%
  \boolfalse{pagetracker}%
  \usebibmacro{prenote}}
{\ifciteindex
  {\indexnames{labelname}}
  {}%
  \printtext[bibhyperref]{\printnames{labelname}}}
{\multicitedelim}
{\usebibmacro{postnote}}

% EN: natbib compatibility
%\newcommand{\citep}[1]{\cite{#1}}
%\newcommand{\citet}[1]{\citeauthor{#1} \cite{#1}}
% EN: Beginning of sentence - analogous to cleveref - important for names such as "zur Muehlen"
%\newcommand{\Citep}[1]{\cite{#1}}
%\newcommand{\Citet}[1]{\Citeauthor{#1} \cite{#1}}

% DE: Blindtext. Paket "blindtext" ist fortgeschritterner als "lipsum" und kann auch Mathematik im Text (http://texblog.org/2011/02/26/generating-dummy-textblindtext-with-latex-for-testing/)
%     kantlipsum (https://www.ctan.org/tex-archive/macros/latex/contrib/kantlipsum) ist auch ganz nett, aber eben auch keine Mathematik
%     Wird verwendet, um etwas Text zu erzeugen, um eine volle Seite wegen Layout zu sehen.
\usepackage[math]{blindtext}


% EN: Make LaTeX logos available by commands. E.g., \lualatex
%     Disabled, because currently causes \not= already defined
%\usepackage{dtk-logos}

% quick replacement:
\newcommand{\LuaLaTeX}{Lua\LaTeX\xspace}
\newcommand{\lualatex}{\LuaLaTeX}

% DE: Neue Pakete bitte VOR hyperref einbinden. Insbesondere bei Verwendung des
%     Pakets "index" wichtig, da sonst die Referenzierung nicht funktioniert.
%     Für die Indizierung selbst ist unter http://xindy.sourceforge.net
%     ein gutes Tool zu erhalten.
%     Hier also neue packages einbinden.
% EN: Add new packages at this place.


% EN: Provides hyperlinks
%     Option "unicode" fixes umlauts in the PDF bookmarks - see https://tex.stackexchange.com/a/338770/9075
%
% DE: Erlaubt Hyperlinks im Dokument.
%     Alle Optionen nach \hypersetup verschoben, sonst crash
%     Siehe auch: "Praktisches LaTeX" - www.itp.uni-hannover.de/~kreutzm
\usepackage[unicode]{hyperref}


% EN: Define colors
% DE: Da es mit KOMA 3 und xcolor zu Problemen mit den global Options kommt MÜSSEN die Optionen so gesetzt werden.
%     Eigene Farbdefinitionen ohne die Namen des xcolor packages
\definecolor{darkblue}{rgb}{0,0,.5}
\definecolor{black}{rgb}{0,0,0}


% EN: Define color of links and more
\hypersetup{
  % have both title and number hyperlinking to content
  linktoc=all,
  bookmarksnumbered=true,
  bookmarksopen=true,
  bookmarksopenlevel=1,
  breaklinks=true,
  colorlinks=true,
  pdfstartview=Fit,
  pdfpagelayout=SinglePage, % DE: Alterntaive: TwoPageRight -- zweiseitige Darstellung: ungerade Seiten rechts im PDF-Viewer - siehe auch http://tex.stackexchange.com/a/21109/9075
  %pdfencoding=utf8, % EN: This is probably the same as passing the option "unicode" at \usepackage{hyperref}
  filecolor=darkblue,
  urlcolor=darkblue,
  linkcolor=black,
  citecolor=black
}


% EN: Abbreviations - has to be loaded after hyperref
% DE: Abkürzungsverzeichnis - muss nach hyperref geladen werden
%
% DE: siehe http://www.dickimaw-books.com/cgi-bin/faq.cgi?action=view&categorylabel=glossaries#glsnewwriteexceeded
\usepackage[acronym,indexonlyfirst,nomain]{glossaries}
\ifdeutsch
  \addto\captionsngerman % DE: siehe https://tex.stackexchange.com/a/154566
  {%
    \renewcommand*{\acronymname}{Abkürzungsverzeichnis}
  }
\else
  \renewcommand*{\acronymname}{List of Abbreviations}
\fi
\renewcommand*{\glsgroupskip}{}
%
% EN: Removed Glossarie as a table as a quick fix to get the template working again
%     See http://tex.stackexchange.com/questions/145579/how-to-print-acronyms-of-glossaries-into-a-table
%
\makenoidxglossaries


% EN: Extensions for references inside the document (\cref{fig:sample}, ...)
% DE: cleveref für cref statt autoref, da cleveref auch bei Definitionen funktioniert
\usepackage[capitalise,nameinlink,noabbrev]{cleveref}
\ifdeutsch
  \crefname{table}{Tabelle}{Tabellen}
  \Crefname{table}{Tabelle}{Tabellen}
  \crefname{figure}{\figurename}{\figurename}
  \Crefname{figure}{Abbildung}{Abbildungen}
  \crefname{equation}{Gleichung}{Gleichungen}
  \Crefname{equation}{Gleichung}{Gleichungen}
  \crefname{theorem}{Theorem}{Theoreme}
  \Crefname{theorem}{Theorem}{Theoreme}
  \crefname{listing}{\lstlistingname}{\lstlistingname}
  \Crefname{listing}{Listing}{Listings}
  \crefname{section}{Abschnitt}{Abschnitte}
  \Crefname{section}{Abschnitt}{Abschnitte}
  \crefname{paragraph}{Abschnitt}{Abschnitte}
  \Crefname{paragraph}{Abschnitt}{Abschnitte}
  \crefname{subparagraph}{Abschnitt}{Abschnitte}
  \Crefname{subparagraph}{Abschnitt}{Abschnitte}
\else
  \crefname{listing}{\lstlistingname}{\lstlistingname}
  \Crefname{listing}{Listing}{Listings}
\fi


% DE: Zur Darstellung von Algorithmen
%     Algorithm muss nach hyperref geladen werden
\usepackage[chapter]{algorithm}
\usepackage[]{algpseudocode}


% DE: Links auf Gleitumgebungen springen nicht zur Beschriftung,
%     Doc: http://mirror.ctan.org/tex-archive/macros/latex/contrib/oberdiek/hypcap.pdf
%     sondern zum Anfang der Gleitumgebung
\usepackage[all]{hypcap}


% DE: Deckblattstyle
%
\ifdeutsch
  \PassOptionsToPackage{language=german}{scientific-thesis-cover}
\else
  \PassOptionsToPackage{language=english}{scientific-thesis-cover}
\fi


% EN: Bugfixes packages
%\usepackage{fixltx2e} %Fuer neueste LaTeX-Installationen nicht mehr benoetigt - bereinigte einige Ungereimtheiten, die auf Grund von Rueckwaertskompatibilitaet beibahlten wurden.
%\usepackage{mparhack} %Fixt die Position von marginpars (die in DAs selten bis gar nicht gebraucht werden}
%\usepackage{ellipsis} %Fixt die Abstaende vor \ldots. Wird wohl auch nicht benoetigt.


% EN: Settings for captions of floats
% DE: Formatierung der Beschriftungen
%
\captionsetup{
  format=hang,
  labelfont=bf,
  justification=justified,
  %single line captions should be centered, multiline captions justified
  singlelinecheck=true
}


% EN: New float environments for listings and algorithms
%
% \floatstyle{ruled} % TODO: enabled or disabled causes no change - listings and algorithms are always ruled
%
\newfloat{Listing}{tbp}{code}[chapter]
\crefname{Listing}{Listing}{Listings}

\newfloat{Algorithmus}{tbp}{alg}[chapter]
\ifdeutsch
  \crefname{Algorithmus}{Algorithmus}{Algorithmus}
\else
  \crefname{Algorithmus}{Algorithm}{Algorithms}
  \floatname{Algorithmus}{Algorithm}
\fi



% EN: Various chapter styles
% DE: unterschiedliche Chapter-Styles
%     u.a. Paket fncychap

% Andere Kapitelueberschriften
% falls einem der Standard von KOMA nicht gefaellt...
% Falls man zurück zu KOMA moechte, dann muss jede der vier folgenden Moeglichkeiten deaktiviert sein.

%\usepackage[Sonny]{fncychap}

%\usepackage[Bjarne]{fncychap}

%\usepackage[Lenny]{fncychap}

%DE: Zur Aktivierung eines der folgenden Möglichkeiten ein Paar von "\iffalse" und "\fi" auskommentieren

\iffalse
  \usepackage[Bjarne]{fncychap}
  \ChNameVar{\Large\sf} \ChNumVar{\Huge} \ChTitleVar{\Large\sf}
  \ChRuleWidth{0.5pt} \ChNameUpperCase
\fi

\iffalse
  \usepackage[Rejne]{fncychap}
  \ChNameVar{\centering\Huge\rm\bfseries}
  \ChNumVar{\Huge}
  \ChTitleVar{\centering\Huge\rm}
  \ChNameUpperCase
  \ChTitleUpperCase
  \ChRuleWidth{1pt}
\fi

\iffalse
  \usepackage{fncychap}
  \ChNameUpperCase
  \ChTitleUpperCase
  \ChNameVar{\raggedright\normalsize} %\rm
  \ChNumVar{\bfseries\Large}
  \ChTitleVar{\raggedright\Huge}
  \ChRuleWidth{1pt}
\fi

\iffalse
  \usepackage[Bjornstrup]{fncychap}
  \ChNumVar{\fontsize{76}{80}\selectfont\sffamily\bfseries}
  \ChTitleVar{\raggedright\Large\sffamily\bfseries}
\fi

% EN: Complete different chapter style - self made

% Innen drin kann man dann noch zwischen
%   * serifenloser Schriftart (eingestellt)
%   * serifenhafter Schriftart (wenn kein zusaetzliches Kommando aktiviert ist) und
%   * Kapitälchen wählen
\iffalse
  \makeatletter
  %\def\thickhrulefill{\leavevmode \leaders \hrule height 1ex \hfill \kern \z@}

  %Fuer Kapitel mit Kapitelnummer
  \def\@makechapterhead#1{%
    \vspace*{10\p@}%
    {\parindent \z@ \raggedright \reset@font
      %Default-Schrift: Serifenhaft (gut fuer englische Dokumente)
      %A) Fuer serifenlose Schrift:
      \fontfamily{phv}\selectfont
      %B) Fuer Kapitaelchen:
      %\fontseries{m}\fontshape{sc}\selectfont
      %C) Fuer ganz "normale" Schrift:
      %\normalfont
      %
      \Large \@chapapp{} \thechapter
      \par\nobreak\vspace*{10\p@}%
      \interlinepenalty\@M
      {\Huge\bfseries\baselineskip3ex
        %Fuer Kapitaelchen folgende Zeile aktivieren:
        %\fontseries{m}\fontshape{sc}\selectfont
        #1\par\nobreak}
      \vspace*{10\p@}%
      \makebox[\textwidth]{\hrulefill}%    \hrulefill alone does not work
      \par\nobreak
      \vskip 40\p@
    }}

  %Fuer Kapitel ohne Kapitelnummer (z.B. Inhaltsverzeichnis)
  \def\@makeschapterhead#1{%
    \vspace*{10\p@}%
    {\parindent \z@ \raggedright \reset@font
      \normalfont \vphantom{\@chapapp{} \thechapter}
      \par\nobreak\vspace*{10\p@}%
      \interlinepenalty\@M
      {\Huge \bfseries %
        %Default-Schrift: Serifenhaft (gut fuer englische Dokumente)
        %A) Fuer serifenlose Schrift folgende Zeile aktivieren:
        \fontfamily{phv}\selectfont
        %B) Fuer Kapitaelchen folgende Zeile aktivieren:
        %\fontseries{m}\fontshape{sc}\selectfont
        #1\par\nobreak}
      \vspace*{10\p@}%
      \makebox[\textwidth]{\hrulefill}%    \hrulefill does not work
      \par\nobreak
      \vskip 40\p@
    }}
  %
  \makeatother
\fi


% DE: Minitoc-Einstellungen
%\dominitoc
%\renewcommand{\mtctitle}{Inhaltsverzeichnis dieses Kapitels}


% EN: Nicer paragraph line placement:
%     - Disable single lines at the start of a paragraph (Schusterjungen)
%     - Disable single lines at the end of a paragraph (Hurenkinder)
%     Normally, this is clubpenalty and widowpenalty, but using a package, it feels more non-hacky
\usepackage[all,defaultlines=3]{nowidow}
%
\displaywidowpenalty = 10000


% EN: Try to get rid of "overfull hbox" things and let text flow batter
%     See also
%       - http://groups.google.de/group/de.comp.text.tex/browse_thread/thread/f97da71d90442816/f5da290593fd647e?lnk=st&q=tolerance+emergencystretch&rnum=5&hl=de#f5da290593fd647e
%       - http://www.tex.ac.uk/cgi-bin/texfaq2html?label=overfull
\tolerance=2000
%
% EN: This could be increased to 20pt
\setlength{\emergencystretch}{3pt}
%
% EN: Suppress hbox warnings if less than 1pt
\setlength{\hfuzz}{1pt}


% EN: Fix names for algorithms in German
% DE: fuer algorithm.sty: - falls Deutsch und nicht Englisch.
\ifdeutsch
  \floatname{algorithm}{Algorithmus}
  \renewcommand{\listalgorithmname}{Verzeichnis der Algorithmen}
\fi


% EN: The euro sign
% DE: Das Euro Zeichen
%     Fuer Palatino (mathpazo.sty): richtiges Euro-Zeichen
%     Alternative: \usepackage{eurosym}
\newcommand{\EUR}{\ppleuro}


% Float-placements - http://dcwww.camd.dtu.dk/~schiotz/comp/LatexTips/LatexTips.html#figplacement
% and http://people.cs.uu.nl/piet/floats/node1.html
\renewcommand{\topfraction}{0.85}
\renewcommand{\bottomfraction}{0.95}
\renewcommand{\textfraction}{0.1}
\renewcommand{\floatpagefraction}{0.75}
%\setcounter{totalnumber}{5}

% EN: ensure that floats covering a whole page are placed at the top of the page
%    see http://tex.stackexchange.com/a/28565/9075
\makeatletter
\setlength{\@fptop}{0pt}
\setlength{\@fpbot}{0pt plus 1fil}
\makeatother



% DE: Bei Gleichungen nur dann die Nummer zeigen, wenn die Gleichung auch referenziert wird
%     Funktioniert mit MiKTeX Stand 2012-01-13 nicht. Deshalb ist dieser Schalter deaktiviert.
%
%\mathtoolsset{showonlyrefs}


% EN: Margins
% DE: Ränder
%     Viele Moeglichkeiten, die Raender im Dokument einzustellen.
%
%     Satzspiegel neu berechnen. Dokumentation dazu ist in "scrguide.pdf" von KOMA-Skript zu finden
%     Optionen werden bei \documentclass[] in ausarbeitung.tex mitgegeben.
% \typearea[current]{current} %neu berechnen, da neue Schrift eingebunden

%\usepackage{a4}
%\usepackage{a4wide}
%\areaset{170mm}{277mm} %a4:29,7hochx21mbreit

%Wer die Masse direkt eingeben moechte:
%Bei diesem Beispiel wird die Regel nicht beachtet, dass der innere Rand halb so gross wie der aussere Rand und der obere Rand halb so gross wie der untere Rand sein sollte
%\usepackage[inner=2.5cm, outer=2.5cm, includefoot, top=3cm, bottom=1.5cm]{geometry}

% EN: Package geometry to enlarge on page
%
%     Normally, geometry should not be used as the typearea package calculates the margins perfectly for printing
%     However, we want better screen-readable documents where the content does not "jump"
%     Thus, we fix the margins left and right to the same value
%
%     Source: http://www.howtotex.com/tips-tricks/change-margins-of-a-single-page/
%
\usepackage[
  left=3cm,right=3cm,top=2.5cm,bottom=2.5cm,
  headsep=18pt,
  footskip=30pt,
  includehead,
  includefoot
]{geometry}


% EN: Provides todo notes
% DE: schoene TODOs
\ifdeutsch
  \usepackage[colorinlistoftodos,ngerman]{todonotes}
\else
  \usepackage[colorinlistoftodos]{todonotes}
\fi
\setlength{\marginparwidth}{2,5cm}

\let\xtodo\todo
\renewcommand{\todo}[1]{\xtodo[inline,color=black!5]{#1}}
\newcommand{\utodo}[1]{\xtodo[inline,color=green!5]{#1}}
\newcommand{\itodo}[1]{\xtodo[inline]{#1}}


% EN: Enable footnotes in tables.
%     This package superseeds the 1997 package "footnote"
\usepackage{footnotehyper}
% TODO: The footnotehyper author recommends to enclose the respective area with \begin{savenotes} ... \end{savenotes}
\makesavenoteenv{tabular}
\makesavenoteenv{table}
% Reuse of footnotes, see http://tex.stackexchange.com/questions/10102/multiple-references-to-the-same-footnote-with-hyperref-support-is-there-a-bett
\crefformat{footnote}{#2\footnotemark[#1]#3}


% EN: pgfplots (optional if the ppackage is installed)
%     PGFPlots draws high-qual­ity func­tion plots in nor­mal or log­a­rith­mic scal­ing
\IfFileExists{pgfplots.sty}{
  \usepackage{pgfplots}
  % EN: highest version supported by overleaf as of 2018-03-16
  \pgfplotsset{compat=1.14}
}{}


% EN: pgfplotstable (optional if the ppackage is installed)
%     PGFPlots generates tables from csv files
\IfFileExists{pgfplotstable.sty}{
  \usepackage{pgfplotstable}
}{}


% EN: Package for creating graphics programmatically
\usepackage{tikz}


% EN: Package for creating uml diagramms
\usepackage{tikz-uml}


% EN: Forest: apgf/TikZ-based package for drawing linguistic trees - https://ctan.org/pkg/forest
\usepackage{forest}


% EN: Enable PlantUML listings in the environment "plantuml"
\IfFileExists{plantuml.sty}{
  \usepackage[output=latex]{plantuml}
}{}


% EN: Layout: bottoms of pages not aligned to each other
% DE: Der untere Rand darf "flattern"
\raggedbottom


% DE: Wie tief wird das Inhaltsverzeichnis aufgeschlüsselt
% 0 --\chapter
% 1 --\section % fuer kuerzeres Inhaltsverzeichnis verwenden - oder minitoc benutzen
% 2 --\subsection
% 3 --\subsubsection
% 4 --\paragraph
\setcounter{tocdepth}{1}


% EN: Fixes wrong spacing in the TOC.
%     Source: https://tex.stackexchange.com/a/33842/9075 -> comment by esdd
\RedeclareSectionCommand[tocnumwidth=2.8em]{section}


% DE: Angaben in die PDF-Infos uebernehmen
\makeatletter
\hypersetup{
  pdftitle={}, %Titel der Arbeit
  pdfauthor={}, %Author
  pdfkeywords={}, % CR-Klassifikation und ggf. weitere Stichworte
  pdfsubject={}
}
\makeatother


% EN: Higher compression of the output PDF
\pdfcompresslevel=9


% EN: Required for recent version of komascript, as some packges are not that compatible with KOMAScript as they should be
%     Has to be loaded at the *very* end, so we use "\AtEndPreamble" by etoolsbox
\usepackage{etoolbox}
\AtEndPreamble{\usepackage{scrhack}}


% EN: Provide tables over multiple pages
\usepackage{longtable}


% EN: Show LaTeX commands and their results in the document
%     Enables the command \PrintDemo
% See https://github.com/latextemplates/scientific-thesis-template/issues/82 for further discussion
\usepackage{latexdemo}


% DE: Fuer deutsche Texte: Weniger Silbentrennung, mehr Abstand zwischen den Woertern
\ifdeutsch
  \setlength{\emergencystretch}{3em} % Silbentrennung reduzieren durch mehr frei Raum zwischen den Worten
\fi

\usepackage{chronology}


\usepackage[
  title={A Tool for the Estimation of Lattice Parameters},
  author={Nicolai Krebs},
  type=bachelor,
  institute=sec, % or other institute names - or just a plain string using {Demo\\Demo...}
  course={Informatik, B.Sc.},
  examiner={Prof.\ Dr.\ Ralf Küsters},
  supervisor={Marc Rivinius,\ M.Sc.},
  startdate={April 22, 2021},
  enddate={October 22, 2021}
]{scientific-thesis-cover}

% Hier stehen alle Abkürzungen
\newacronym{er}{ER}{error rate}
\newacronym{fr}{FR}{Fehlerrate}
\newacronym[plural={RDBMS},shortplural={RDBMS}]{rdbms}{RDBMS}{Relational Database Management System}


\makeindex

\begin{document}

%tex4ht-Konvertierung verschönern
\iftex4ht
  % tell tex4ht to create picures also for formulas starting with '$'
  % WARNING: a tex4ht run now takes forever!
  %\Configure{$}{\PicMath}{\EndPicMath}{}
  %$ % <- syntax highlighting fix for emacs
  \Css{body {text-align:justify;}}

  %conversion of .pdf to .png
  \Configure{graphics*}
  {pdf}
  {\Needs{"convert \csname Gin@base\endcsname.pdf
      \csname Gin@base\endcsname.png"}%
    \Picture[pict]{\csname Gin@base\endcsname.png}%
  }
\fi

%\VerbatimFootnotes %verbatim text in Fußnoten erlauben. Geht normalerweise nicht.

% DE: wird fuer Tabellen benötigt (z.B. >{centering\RBS}p{2.5cm} erzeugt einen zentrierten 2,5cm breiten Absatz in einer Tabelle
\newcommand{\RBS}{\let\\=\tabularnewline}

% EN: To avoid issues with Springer's \mathplus
%     See also http://tex.stackexchange.com/q/212644/9075
\providecommand\mathplus{+}

% DE: typoraphisch richtige Abkürzungen
\newcommand{\zB}{z.\,B.\xspace}
\newcommand{\bzw}{bzw.\xspace}
\newcommand{\usw}{usw.\xspace}
\renewcommand{\dh}{d.\,h.\xspace}

% EN: from hmks makros.tex - \indexify
\newcommand{\toindex}[1]{\index{#1}#1}

% DE: Tipp aus "The Comprehensive LaTeX Symbol List"
\newcommand{\dotcup}{\ensuremath{\,\mathaccent\cdot\cup\,}}

% DE: Anstatt $|x|$ $\abs{x}$ verwenden.
%     Die Betragsstriche skalieren automatisch, falls "x" etwas größer sein sollte...
\newcommand{\abs}[1]{\left\lvert#1\right\rvert}

% DE: für Zitate
\newcommand{\citeS}[2]{\cite[S.~#1]{#2}}
\newcommand{\citeSf}[2]{\cite[S.~#1\,f.]{#2}}
\newcommand{\citeSff}[2]{\cite[S.~#1\,ff.]{#2}}
\newcommand{\vgl}{vgl.\ }
\newcommand{\Vgl}{Vgl.\ }

% EN: For the algorithmic package
\newcommand{\commentchar}{\ensuremath{/\mkern-4mu/}}
\algrenewcommand{\algorithmiccomment}[1]{\hfill $\commentchar$ #1}

% DE: Seitengrößen - Gegen Schusterjungen und Hurenkinder...
\newcommand{\largepage}{\enlargethispage{\baselineskip}}
\newcommand{\shortpage}{\enlargethispage{-\baselineskip}}

\newcommand{\initialism}[1]{%
  \ifdeutsch%
    \textsc{#1}\xspace%
  \else%
    \textlcc{#1}\xspace%
  \fi%
}
\newcommand{\OMG}{\initialism{OMG}}
\newcommand{\BPEL}{\initialism{BPEL}}
\newcommand{\BPMN}{\initialism{BPMN}}
\newcommand{\UML}{\initialism{UML}}

\pagenumbering{arabic}
\Titelblatt

%Eigener Seitenstil fuer die Kurzfassung und das Inhaltsverzeichnis
\deftriplepagestyle{preamble}{}{}{}{}{}{\pagemark}
%Doku zu deftriplepagestyle: scrguide.pdf
\pagestyle{preamble}
\renewcommand*{\chapterpagestyle}{preamble}



%Kurzfassung / abstract
%auch im Stil vom Inhaltsverzeichnis
\ifdeutsch
  \section*{Kurzfassung}
\else
  \section*{Abstract}
\fi

<Short summary of the thesis>

\cleardoublepage


% BEGIN: Verzeichnisse

\iftex4ht
\else
  \microtypesetup{protrusion=false}
\fi

%%%
% Literaturverzeichnis ins TOC mit aufnehmen, aber nur wenn nichts anderes mehr hilft!
% \addcontentsline{toc}{chapter}{Literaturverzeichnis}
%
% oder zB
%\addcontentsline{toc}{section}{Abkürzungsverzeichnis}
%
%%%

%Produce table of contents
%
%In case you have trouble with headings reaching into the page numbers, enable the following three lines.
%Hint by http://golatex.de/inhaltsverzeichnis-schreibt-ueber-rand-t3106.html
%
%\makeatletter
%\renewcommand{\@pnumwidth}{2em}
%\makeatother
%
\tableofcontents

% Bei einem ungünstigen Seitenumbruch im Inhaltsverzeichnis, kann dieser mit
% \addtocontents{toc}{\protect\newpage}
% an der passenden Stelle im Fließtext erzwungen werden.

\listoffigures
\listoftables

%Wird nur bei Verwendung von der lstlisting-Umgebung mit dem "caption"-Parameter benoetigt
%\lstlistoflistings
%ansonsten:
\ifdeutsch
  \listof{Listing}{Verzeichnis der Listings}
\else
  \listof{Listing}{List of Listings}
\fi

%mittels \newfloat wurde die Algorithmus-Gleitumgebung definiert.
%Mit folgendem Befehl werden alle floats dieses Typs ausgegeben
\ifdeutsch
  \listof{Algorithmus}{Verzeichnis der Algorithmen}
\else
  \listof{Algorithmus}{List of Algorithms}
\fi
%\listofalgorithms %Ist nur für Algorithmen, die mittels \begin{algorithm} umschlossen werden, nötig

% Abkürzungsverzeichnis
\printnoidxglossaries

\iftex4ht
\else
  %Optischen Randausgleich und Grauwertkorrektur wieder aktivieren
  \microtypesetup{protrusion=true}
\fi

% END: Verzeichnisse


% Headline and footline
\renewcommand*{\chapterpagestyle}{scrplain}
\pagestyle{scrheadings}
\pagestyle{scrheadings}
\ihead[]{}
\chead[]{}
\ohead[]{\headmark}
\cfoot[]{}
\ofoot[\usekomafont{pagenumber}\thepage]{\usekomafont{pagenumber}\thepage}
\ifoot[]{}


%% vv  scroll down for content  vv %%































%%%%%%%%%%%%%%%%%%%%%%%%%%%%%%%%%%%%%%%%%%%%%%%%%%%%%%%%%%%%%%%%%%%%%%%%%%%%%%
%
% Main content starts here
%
%%%%%%%%%%%%%%%%%%%%%%%%%%%%%%%%%%%%%%%%%%%%%%%%%%%%%%%%%%%%%%%%%%%%%%%%%%%%%%


\chapter{Introduction}

- rise of quantum computing (short history)

  * conceptual

  * reality


- problem: some hard classical problems no longer hard

  * Shor's Algorithm (Peter Shor, 1994) %Sho97
    => quantum computers can solve the factoring and the discrete logarithm problem in polynomial time

  * application to encryption

  * overview of current encryption methods that will become insecure


- one solution (among hash-based, code-based, isogeny-based, and multivariate): lattice crypto

  * overview over history and capability of lattice crypto

  * advantages: good (quasilinear) asymptotic key sized, good concrete runtimes and key sizes, worst-case secure instantiations, advanced cryptographic primitives previously infeasible

  * including intro to LWE/SIS and applications to build crypto systems

    . SIS: signature schemes, hash functions

    . LWE: ``cryptomania'' applications (PKE, ...), signature schemes, lines:

      - cryptographic applications

      - establishing theoretical and asymptotic hardness \cite{Reg05} % ? Check
        \cite{BLPRS13, MP13} 
      - concrete hardness of LWE: attacks, runtime estimates, 

  * briefly outline concept and benefits of hard-case to average-case reductions


- purpose of this thesis

  * building schemes: need realistic hardness estimates of schemes for given parameter settings

  * lack in the past: no unified/easy to use tool => thesis aims to solve this problem
    tool we call \textit{Lattice Parameter Estimation} 
    LWE instances are estimated by calling various estimation functions from the LWE Estimator \cite{APS15}, which we will refer to as \textit{Estimator}.


- overview of chapters/how to read



\chapter{Preliminaries} \label{chap:Preliminaries}

\section{Notation}
In the following, we denote vectors by bold lower-case letters like $\mathbf{v}$ and matrices by bold upper-case letters $\mathbf{M}$. We interchangably use matrix notation and sets of column vectors $\left[\mathbf{v}_1 \cdots \mathbf{v}_n\right] = \left\{\mathbf{v}_1, \dots, \mathbf{v}_n\right\}$. Unless specified otherwise, by $\| \cdot \|$ or simply \textit{norm} we refer to the Euclidean norm. By $[n]$ we denote the set $\{1, \dots, n\}$ for $n\in \mathbb{Z}^+$.  
% TODO: anything else? Inner product, negation in ring


\section{Math} 
% TODO: check out MG02 for more details for this section!!!

% TODO: some algebraic background?
\subsection{Norms and Bounds} % TODO: check
%TODO: write some prose about why we need that
% TODO: make sure that it is clear that these are bounds!!!
Let $\mathcal{R}_q$ be a ring as defined in \cite{BDLOP18} and $f \in \mathcal{R}_q$ with $f = \sum_i f_i X^i$. We define the following norms \cite{BDLOP18}:
\begin{align}
  \ell_1 : \| f| \|_1 &= \sum_i |f_i|\\
  \ell_2 : \| f| \|_2 &= \left(\sum_i |f_i|^2\right) ^{\frac{1}{2}}\\
  \ell_\infty : \| f| \|_\infty &= \max_i |f_i|
\end{align}

Then the following inequations hold \cite{BDLOP18}:
\begin{align}
  \| f \|_1 &\leq \sqrt{n} \| f \|_2 \label{norm1}\\
  \| f \|_1 &\leq n \| f \|_\infty \label{norm2}\\
  \| f \|_2 &\leq \sqrt{n} \| f \|_\infty \;\;(\text{since }  \sqrt{n} \| f \|_2 \leq n \| f \|_\infty) \label{norm3}\\
  \| f \|_\infty& \leq \| f \|_1 \label{norm4}
\end{align}

Let $\mathcal{O}_K$ be the ring of integers of a number field $K=\mathbb{Q}(\theta)$, where $\theta$ is an algebraic number and $\sigma$ denote the canonical embedding as defined in \cite{DPSZ12}. Then, for $x, y \in \mathcal{O}_K$ it holds the following inequations hold (we assume that $C_m$ in \cite{DPSZ12} is $1$) \cite{DPSZ12}. 
\begin{align}
  \| f \|_\infty &\leq \| \sigma(f) \|_\infty \label{norm5}\\
  \| \sigma(f) \|_\infty &\leq \| f \|_1 \label{norm6}
\end{align}

From the above inequations, we obtain the following norm transformations to $\ell_p$-norms:
\begin{itemize}
  \item From \cref{norm1}, it follows that $\| f \|_1 \leq \sqrt{n} \| f \|_2$ and from \cref{norm2}, $\| f \|_1 \leq n \| f \|_\infty$.
  \item From \cref{norm3} and \cref{norm4}, it follows that $\| f \|_2 \leq \sqrt{n}  \| f \|_1$ and from \cref{norm3}, $\| f \|_2 \leq \sqrt{n}  \| f \|_\infty$.
  \item From \cref{norm4}, it follows that $\| f \|_\infty \leq  \| f \|_1$ and from \cref{norm1} and \cref{norm4}, $\| f \|_\infty \leq \sqrt{n}  \| f \|_2$.
  \item From \cref{norm6}, it follows that $\| \sigma(f) \|_\infty \leq  \| f \|_1$, from \cref{norm1} and \cref{norm6}, $\| \sigma(f) \|_\infty \leq \sqrt{n}  \| f \|_2$, and from \cref{norm2} and \cref{norm6}, $\| \sigma(f) \|_\infty \leq n  \| f \|_\infty$.
\end{itemize}

Likewise, we get the following transformations to the $\mathcal{C}_\infty$-norm:
\begin{itemize}
  \item From \cref{norm2} and \cref{norm5}, it follows that $\| f \|_1 \leq  n \| \sigma(f) \|_\infty$.
  \item From \cref{norm3} and \cref{norm5}, it follows that $\| f \|_2 \leq  \sqrt{n} \| \sigma(f) \|_\infty$.
  \item From \cref{norm5}, it follows that $\| f \|_\infty \leq  \| \sigma(f) \|_\infty$.
\end{itemize}

Let $f$ be defined as above and let $g \in \mathcal{R}_q$ where $g = \sum_i \overline{g}_i X^i$ where $g_i \in \left[-(q-1)/2, (q-1)/2\right]$ and $\overline{g}_i = g_i \mod q$ as in \cite{BDLOP18}. Then, we can define the following inequations for multiplication according to \cite{BDLOP18}:

\begin{itemize}
  \item If $\|f\|_\infty \leq \beta, \|g\|_1 \leq \gamma$ then $\|f \cdot g\|_\infty \leq \beta \cdot \gamma$.
  \item If $\|f\|_2 \leq \beta, \|g\|_2 \leq \gamma$ then $\|f \cdot g\|_\infty \leq \beta \cdot \gamma$.
\end{itemize}

Let $x, y \in \mathcal{O}_K$. Again, we assume that $C_m = 1$. Then, the following inequation holds according to \cite{DPSZ12}:
\begin{align}
  \| x \cdot y \|_\infty \leq C_m \cdot n^2 \cdot \| x \|_\infty \cdot \| y \|_\infty\\
  \| \sigma(x \cdot y) \|_\infty \leq  \| \sigma(x) \|_\infty \cdot \| \sigma(y) \|_\infty.
\end{align}



\subsection{Lattices}
- background and history: example from lecture -> change % TODO: from lecture -> change!!!

  * Birhoff [Bir40]

  * cryptoanalysis [LLL82]

  * cryptosystems [Ajt96, HPS98]
    SIS introduced Ajtai \cite{Ajt96}

  * [MR04]

  * LWE, assumption: worst-case lattice problems are hard [Reg05]

  * fully homomorphic [Gen09]

  * BGV scheme [BV11, BGV12]

  * tools [LPR10, LPR13] ideal latties, RLWE

Other Notes: %TODO
- PKE \cite{AD97, Reg03, Reg05}, CCA security \cite{PW08, Pei09}, identity-based encryption \cite{GPV08, CHKP10, ABB10}, fully homomorphic \cite{Gen09a}
- , LWE introduced by \cite{Reg05} "provably as hard as certain lattice problems in worst case, appear to require time exponential in main security parameter to solve"
- NTRU \cite{HPS98}
- $q$-ary lattice: modulus $q\geq 2$

- math % TODO: rewrite or \cite{Gop16}
  * lattice $\Lambda$ % TODO: check all n/m's

    

    - discrete additive subgroup of $\mathbb{R}^m$ % define discrete and additive subgroup?

    - Let $\mathbf{b}_1, \ldots, \mathbf{b}_n \in \mathbb{R}^m$ be a set of linearly independent basis vectors and $\mathbf{B} = \left[\mathbf{b}_1, \ldots, \mathbf{b}_n\right] \in \mathbb{R}^{m\times n}$ be the corresponding basis with column vectors $\mathbf{b}_i$

    - $n$ is the dimension of the Lattice

    - $\Lambda(\mathbf{B})$ defined by all integer combinations of elements of $\mathbf{B}$:
    \begin{equation}
      \Lambda(\mathbf{B}) = \left\{ x \in \mathbb{R}^m \mid \exists \alpha_1, \ldots, \alpha_n \in \mathbb{Z} : \mathbf{x} = \sum_{i=1}^n \alpha_i \mathbf{b}_i \right\}
    \end{equation} 
    - show example plot % TODO

    - full-ranked lattice: dimension is maximal, $m$

    - basis $\mathbf{B}$ is not unique -> let $\mathbf{U}\in \mathbb{Z}^{n\times n}$ be a modular matrix (determinant is $\pm1$), then $\mathbf{B}\cdot \mathbf{U}$ is also a basis of the $\Lambda$ ($\mathbf{U} \cdot \mathbb{Z}^{n} = \mathbb{Z}^{n}$) -> different basis for the same lattice $\Lambda$

    - lattice coset:
      quotient group $\mathbb{R}^n/\Lambda$ of cosets
      \begin{equation*}
        \mathbf{c} + \Lambda = {\mathbf{c} + \mathbf{v} \mid v \in \Lambda}
      \end{equation*}
      with $\mathbf{c} \in \mathbb{R}^n$ % TODO: maybe add plot

    - fundamental domain: subset of $\mathbb{R}^m$ containing exactly one representative of every coset

    - (shifted) fundamental parallelipiped \label{eq:fundamental-parallelipiped}: $\mathcal{P}(\mathbf{B}) = \mathbf{B} \cdot [ - 1/2, 1/2)^n = \left\{ \mathbf{x} \in \mathbb{R}^m \mid \mathbf{x} \sum_{i=1}^n \gamma_i \mathbf{x}_i, \gamma_i \in  [ - 1/2, 1/2) \right\}$ every coset has representative  % c - B * \left\lfloor^{-1} \cdot c\right\rceil
    % TODO: first define [0, 1)
    % intuition: collection of all points that can be written as By where y in [0, 1)^n
    % TODO: maybe add plot

    - determinant of lattice $\Lambda(\mathbf{B})$: $\sqrt{\text{det}\left(\mathbf{B}^\intercal \mathbf{B}\right)}$. For a full-ranked lattice the determinant is 
    \begin{equation}
      \det(\Lambda(\mathbf{B})) = |\det(\mathbf{B})|
    \end{equation}
      is well-defined (independent from basis) => volume of fundamental domain % TODO define
      can be generalized to not full-ranked => $\det(\Lambda(\mathbf{A})) = \sqrt{\det(\mathbf{A}^\perp \mathbf{A})}$ % TODO perp

  * minimum distance of $\lambda_1(\Lambda)$ of a lattice is the length of its shortest nonzero vector, i.e. $\lambda_1(\Lambda) \min_{v \in \Lambda \setminus \{0\}}$
  * $i$th successive minimum $\lambda_i(\Lambda)$
    - Let $r\in \mathbb{R}$ and $\mathbf{c} \in \mathbb{R}^m$, then we define $\mathcal{B}(\mathbf{c}, r)$ as the ball of radius $r$ with center $\mathbf{c}$. 

    - smallest radius $r$ such that the ball $\mathcal{B}(\mathbf{0}, r)$ centered at the origin of $\Lambda$ contains $i$ linearly independent lattice vectors. 

    - in general hard to calculate $\lambda_i(\Lambda(\mathbf{B}))$ for a given basis  
  % * Gaussian heuristic: 
  %   - estimate length of shortest lattice vactor in $n$-dimensional lattice $Lambda$, i.e. estimate $\lambda_1(\Lambda)$ given $\det{\Lambda}$
  %   \begin{equation}
  %     \lambda_1{\Lambda} \approx \frac{\Gamma(1 + n/2)^{1/n}}{\sqrt{\pi}} \det(\Lambda)^{1/n}
  %   \end{equation}
  
  * modular integer (or $q$-ary) lattices % \cite{MR09}

    - full-ranked lattice $\Lambda$ such that $q\mathbb{Z}^m \subseteq	\Lambda \subseteq	\mathbb{Z}^m$ given $q \in \mathbb{N}$ => if $\mathbf{x} \in \mathbb{Z}^m$ in $\Lambda$ then $\mathbf{x} \mod q$ also in $\Lambda$. 

    - can be specified in two ways by matrix $\mathbf{A} \in \mathbb{Z}_q^{m\times n}$: % TTODO switch m and n for consistency
    \begin{equation}
      \Lambda_q(\mathbf{A}) = \left\{ x \in \mathbb{Z}^m \mid \exists y \in \mathbb{Z}^n : \mathbf{x} = \mathbf{A}\mathbf{y} \mod q \right\}
    \end{equation}
    % TODO: corresponds to linear code generated by columns of A (mod q)
    or
    \begin{equation}
      \Lambda_q^\perp(\mathbf{A}) = \left\{ x \in \mathbb{Z}^m \mid  \mathbf{A}^\intercal\mathbf{x} = 0 \mod q \right\}
    \end{equation}

    - finding a short vector in $\Lambda_q(\mathbf{A})$ corresponds to LWE % TODO check

    - finding short vectors in $\Lambda_q^\perp(\mathbf{A})$ corresponds to SIS % TODO check \cite{Reg10}

    - easy to find basis of $\Lambda_q(\mathbf{A})$ \cite{AFG13}

    - with high probability determinant of $q$-ary lattice is $\det(\Lambda_q(\mathbf{A}))=q^{m-n}$ if $\mathbf{A} \in \mathbb{Z}_q^{m\times n}$


  * Gram-Schmidt basis

    - set of column vectors $\mathbf{B} \in \mathbb{Z}_q^{m\times n}$, $\pi_{\text{span}(\mathbf{B})}(\mathbf{t})$ for projection of vector $\mathbf{t}$ unto span of vectors of $\mathbf{B}$

    - $\pi_{\text{span}(\mathbf{B})}(\mathbf{t}) = \mathbf{B}(\mathbf{B}^\perp \mathbf{B})^{-1}\mathbf{B}^\intercal \cdot \mathbf{t}$

    - Gram-Schmidt orthogonalization $\tilde{\mathbf{B}} = \left[\tilde{\mathbf{b}}_1 \cdots \tilde{\mathbf{b}}_n\right]$ of basis $\mathbf{B}$: $\tilde{\mathbf{b}}_i = \mathbf{b}_i - \pi_{\text{span}(\mathbf{b}_1, \ldots, \mathbf{b}_{i-1})}(\mathbf{b}_i)$ for $i \in \{1, \ldots, n\}$ 

    - Gram-schmidt coefficients $\mu_{i, j} = \frac{\left\langle \tilde{\mathbf{b}}_j, \mathbf{b}_i\right\rangle}{\left\langle \tilde{\mathbf{b}}_j, \tilde{\mathbf{b}}_j\right\rangle}$ %Pla18

    Alternative:
    Let $\mathbf{B} = \left[\mathbf{b}_1 \cdots \mathbf{b}_n\right], \mathbf{b}_i \in \mathbb{Z}_q^{m}$ be a basis. Define $\tilde{\mathbf{b}}_i$ as follows: $\tilde{\mathbf{b}}_1 = \mathbf{b}_1$. For $i \in \{2, \ldots, n\}$ let $\tilde{\mathbf{b}}_i$ be the component of $\mathbf{b}_i$ that is orthogonal to the span of $\left\{\mathbf{b}_1, \ldots, \mathbf{b}_{i-1}\right\}$. Then,  $\tilde{\mathbf{B}} = \left[\tilde{\mathbf{b}}_1 \cdots \tilde{\mathbf{b}}_n\right]$ is called the Gram-Schmidt orthogonalization of basis $\mathbf{B}$ where $\| \tilde{\mathbf{b}}_i\| \leq \| \mathbf{b}_i\|$. 
    % TODO Also check out algorithm from AGVW17

  
  * dual of a lattice is "the set of points whose inner products with the vectors in the lattice are integers" $\Lambda$: $\Lambda^{\perp} := \{ \mathbf{y} \in \mathbb{R}^m \mid \forall \mathbf{v} \in \Lambda: \langle \mathbf{y}, \mathbf{v} \rangle \in \mathbb{Z}\}$
    % TODO: show examples
    scaled-by-$q$ dual lattice:  $\{ \mathbf{y} \in \mathbb{Z}^m \mid \forall \mathbf{v} \in \Lambda: \langle \mathbf{y}, \mathbf{v} \rangle = 0 \mod q\}$ % not sure if needed
    basis of the dual of a lattice with basis $\mathbf{B}$ is $\mathbf{B'} = \mathbf{B} (\mathbf{B}^\intercal \mathbf{B})^{-1}$ % https://cseweb.ucsd.edu/classes/wi12/cse206A-a/LecDual.pdf

  * smoothing lemma

  * Voronoi region % \cite{GSJ15}
  The fundamental Voronoi region $\mathcal{V}$ is defined as 
  \begin{equation}
    \mathcal{V} = \left\{ \mathbf{x} \in \mathbb{R}^n \mid \forall \mathbf{y} \in \Lambda : \| \mathbf{x} \| \leq \| \mathbf{x} - \mathbf{y} \| \right\}
  \end{equation}

  * Linear Code \cite{VanLint12} % TODO maybe move to Coded BKW
    Let $\mathbb{F}_q^n$ be the $n$-dimensional vector space over the field $\mathbb{F}_q$. A $q$-ary linear code $\mathcal{C}$ or $[n, k]$-code is a $k$-dimensional linear subspace of $\mathbb{F}_q^n$ such that 
    \begin{itemize}
      \item $\mathbf{0} \in \mathcal{C}$,
      \item if $\mathbf{x}, \mathbf{y} \in \mathcal{C}$, then $\mathbf{x} + \mathbf{x} \in \mathcal{C}$,
      \item and if $\mathbf{x} \in \mathcal{C}$ and $\gamma \in \mathbb{F}_q$, then $\gamma \mathbf{x} \in \mathcal{C}$.
    \end{itemize}
    There are $q^k$ different codewords in $\mathcal{C}$.

    Let $\mathcal{C}$ be a $q$-ary linear $[n, k]$-code. The lattice over $\mathcal{C}$ is defined as % \cite{GJS15}
    \begin{equation}
      \Lambda(\mathcal{C}) = \left\{ \mathbf{x} \in \mathbb{R}^n \mid \exists \mathbf{y} \in \mathcal{C} : \mathbf{x} = \mathbf{y} \mod q  \right\}.
    \end{equation} % TODO check
    Similarly, for a lattice $\Lambda(\mathbf{B})$ a lattice code $\mathcal{C}$ defined by $\Lambda(\mathbf{B})$ and a shaping region $\mathcal{V} \subset \mathbb{R}^n$ (e.g. the Voronoi region) is a subspace of $\mathbb{R}^n$ such that all codewords are lattice vectors in $\Lambda(\mathbf{B})$ within the region $\mathcal{V}$ \cite{SFS08}:
    \begin{equation}
      C' = \left\{ x \in \Lambda(\mathbf{B}) \mid x \in \mathcal{V} \right\}.
    \end{equation} % TODO check


% TODO: where to put

    We define $\text{dist}(\mathbf{t}, \Lambda(\mathbf{B}))$ where $\Lambda(\mathbf{B}) \subset \mathbb{R}^m$ as the distance of a vector $\mathbf{t} \in \mathbb{R}^m$ to the closest lattice vector $\mathbf{v} \in \Lambda(\mathbf{B})$, i.e. 
    \begin{equation}\label{eq:dist}
      \text{dist}(\mathbf{t}, \Lambda(\mathbf{B})) = \min_{\mathbf{v} \in \Lambda(\mathbf{B})}\|\mathbf{t} -  \mathbf{v}\|.
    \end{equation}

    
    

- Lattice problems

  * Minkowski theorem: Let $\Lambda$ be a lattice of dimension $n$, then $\lambda_1 \leq \sqrt{n} \cdot (\det \Lambda)^{\frac{1}{n}}$

  * Lattice reduction: find short basis compared to $\lambda_1(\Lambda)$... 

  * SVP: given a basis $\mathbf{B}$ of lattice $\Lambda$ find shortest nonzero lattice vector =>  $v\in \Lambda$ s.t. $\| v \| = \lambda_1(\Lambda)$

\begin{definition}[$\gamma$-approximate Shortest Vector Problem (SVP$_\gamma$)] \label{def:gammaSVP} % TODO find source
  Given a basis $\mathbf{B}$ of lattice $\Lambda$, find a short lattice vector $v\in \Lambda$ such that $0 < \| v \| \leq \gamma \lambda_1(\Lambda)$
\end{definition}

\begin{definition}[$\kappa$-approximate Hermite Shortest Vector Problem (HSVP$_\kappa$)] \label{def:kappaHSVP} % TODO find source
  Given a basis $\mathbf{B}$ of a lattice $\Lambda(\mathbf{B}) \in \mathbb{R}^m$, find a nonzero lattice vector $\mathbf{v} \in \Lambda$ such that $\| \mathbf{v} \| \leq \kappa \cdot \text{det}(\Lambda)^{\frac{1}{n}}$. 
\end{definition}

  * \textsc{GapSVP}$_\gamma$ (decision version of \textsc{SVP}): "given basis $\mathbf{B}$ of $n$-dimensional lattice $\Lambda$ with either $\lambda_1{\Lambda} \leq 1$ or $\lambda_1{\Lambda} \geq \gamma(n)$, decide which is the case" %\cite{Pei16}
    NP hard for any constant $\gamma$ % Kho04, HR07 see LM09
    fastest algorithm for $1\leq \gamma \leq \text{poly}(n)$ has runtime complexity of $2^{O(n)}$
  
  * $\gamma$-unique Shortest Vector Problem (uSVP$_\gamma$) \cite{LM09}: given lattice $\Lambda$ such that $\lambda_2(\Lambda) > \gamma \lambda_1(\Lambda)$, find shortest nonzero vector in $\mathbf{v} \in \Lambda$ with $\|\mathbf{v}\| = \lambda_1(\Lambda)$
    % or more relaxed version $\|\mathbf{v}\| = \gamma\lambda_1(\Lambda)$
    %NP hard when $\gamma = 1 + 2^{-n^c}$ for some constant $c$ \cite{KS01}

  * CVP$_\gamma$: given  basis $\mathbf{B}$ of $n$-dimensional lattice $\Lambda$ and target $\mathbf{t}\in\mathbb{R}^n$ find point in lattice that is close to $\mathbf{t}$ => find  $\mathbf{v} \in \mathbb{R}^n$ with $\|\mathbf{t} - \mathbf{v}\| < \gamma \min_{\mathbf{v}' \in \Lambda} \|\mathbf{v}' - \mathbf{v}\|$

  * \textsc{SIVP} (shortest independent vector problem): given  basis $\mathbf{B}$ of $n$-dimensional lattice $\Lambda$, find $n$ linearly independent lattice vectors $\mathbf{v}_1, \ldots, \mathbf{v}_n \in \Lambda(\mathbf{B})$ such that $\max_i \|\mathbf{v}_i\|$ for $i \in \{1, \ldots, n\}$ is minimal

  * $\gamma$-Bounded Distance Decoding (BDD$_\gamma$): Given a lattice $\Lambda$ and a target vector $\mathbf{t}$ such that $\text{dist}(\mathbf{t}, \Lambda) < \gamma \lambda_1(\Lambda)$, find the closest lattice vector $\mathbf{v} \in \Lambda$, i.e. find $\mathbf{v} = \min_{\mathbf{v}' \in \Lambda} \|\mathbf{v}' - \mathbf{t}\|$ minimal \cite{LM09}
    % or more relaxed version: find closest vector $\mathbf{v} \in \Lambda$ such that $\|\mathbf{v} - \mathbf{t}\| < \gamma \lambda_1(\Lambda)$
    % alternate NP hard for $\gamma > 1/\sqrt{2}$ \cite{LLM06}

  % TODO: add some kind of discussion, e.g. see 1.2 in LM09 or better overview from bootcamp

  * ideal lattice (do I need that?)

  * ...?

  * eher die Sachen für LWE/SIS als die Sachen für Algorithmen (analog Vorlesung), evtl. 
  
  Intuition für die anderen Sachen...



\subsection{Distributions}

  - Gaussian, def, component-wise, trafo to bound % TODO: GPV08, or LS15

    * definition:
      discrete Gaussian distribution over $q$-ary lattice $\Lambda$ with Gaussian width parameter $s > 0$ and center $\mathbf{c}$, denoted by $D_{\Lambda, s, \mathbf{c}}$: probability of sampling a vector $\mathbf{x}\in \Lambda$ is proportional to $e^{-\pi \|\mathbf{x} - \mathbf{c}\|^2/s^2}$ % \cite{Reg10, GPV08} or "probability distribution that assigns mass proportional to ... to each point $\mathbf{x}\in \Lambda$"
      In order to avoid confusion, throughout this work and in the \textit{Lattice Parameter Estimation} we use $\sigma$ to denote the standard deviation, where $\sigma = \frac{s}{\sqrt{2 \pi}}$, and define $\alpha := \frac{s}{q} = \frac{\sqrt{2\pi} \sigma}{q}$. 

    * better definition in GPV08 => different definition needed for LWE??? % TODO
    * how to do this? => variant of Babai's ``nearest-plane'' algorithm, see \cite{GPV08} % TODO

    * component-wise
    

    For some applications, we receive a Gaussian distribution as input, but require a bound in some norm in order to estimate the hardness of SIS. Hence, we need to transform the Gaussian width parameter into a bound $\beta$ given some security parameter $\text{sec}$. Note that a $n$-dimensional Gaussian can be sampled by sampling $n$ independent 1-dimensional Gaussians. %TODO reformulate
        
    For a Gaussian distribution, the following holds: % put as theorem, also for norm inequalities
    
    \begin{equation}
      \text{Pr}\left[ |X| \geq \beta \right] \leq 2 e^{-\pi \beta^2/s^2}
    \end{equation}
    
    We demand $2 e^{-\pi \beta^2/s^2} \approx 2^{-sec}$, hence
    
    \begin{align*}
      2 e^{-\pi \beta^2/s^2} &\approx 2^{-sec}\\
      -\pi \frac{\beta^2}{s^2} &\approx (-sec - 1)\ln (2)\\
      \beta  &\approx s \sqrt{\frac{(sec + 1) \ln(2)}{\pi}}
    \end{align*}


    * smoothing factor here?

    * Uniform (stuff I use in tool)









\section{Two Important Problems}
Applications: SIS can be used for one-way functions and collision-resistant hasing. LWE can be used to build pseudo-random number generators, public-key encryption schemes and oblivious transfer and secure MPC. Lattice Trapdoors (trapdoor functions, digital signatures)? Punctured Trapdoors (identity-based encryption, attribute-based encryption, predicate encryption)? % TODO: from https://www.youtube.com/watch?v=LlPXfy6bKIY see below for more detail

\subsection{Learning with Errors (LWE)} \label{sec:lwe}
Following based on \cite{Reg10}:% TODO: change or quote 

Introduced by Regev in \cite{Reg09}
Origin: work of Ajtai and Dwork \cite{AD97}, first public-key cryptosystem based on worst-case lattice problems, simlifications/improvements \cite{GGH97b, Reg03} imply hardness result for LWE. 
Early work: hardness based on unique-SVP, Peikert \cite{Pei09} and Lyubashevsky and Micciancio \cite{LM09} show that unique-SVP is essentially equivalent to \textsc{GapSVP}.

- ´cryptomania´ applications: public-key encryption schemes under chosen-plaintext attacks \cite{Reg05, KTX07, PVW08}, and chosen-ciphertext attacks \cite{PW08, Pei09}, oblivious transfer protocoles \cite{PVW08}, identity-based encryption (IBE) schemes \cite{GPV08, CHKP10, ABB10}, leakage-resilient encryption \cite{AGV09, ACPS09, DGKPV10, GKPV10}, and more % TODO change or \cite{Reg10}

- most important: fully homomorphic encryption schemes \cite{Gen09a, BV11, Bra12, GSW13} % TODO 

Intuition: 
- "recover $\mathbf{s} \in \mathbb{Z}_q^n$ given sequence of ´approximate´ random linear equations on $\mathbf{s}$" 
- public matrix  $\mathcal{A} \in \mathbb{Z}^{n \times m}$, secret vector $\mathcal{s}\in \mathbb{Z}^n$, given $\mathcal{z} = \mathcal{A}^\intercal \mathcal{s}$ we can find $\mathbf{s}$ by linear algebra
when we add a small error vector $\mathbf{e} \in \mathbb{Z}^m$, solving $\mathcal{z}' = \mathcal{A}^\intercal \mathcal{s} + \mathbf{e}$ for $\mathbf{s}$ or distinguishing $\mathbf{z}'$ from uniform becomes hard


Formal Definition: 
\begin{definition}[LWE Distribution \cite{Reg10}] %TODO
  For $n \geq 1$, modulus $q \geq 2$, error distribution $\chi$ on $\mathbb{Z}_q$, and a fixed secret vector $\mathbf{s}$, let $\mathcal{A}_{\mathbf{s}, \chi}$ be the probability distribution over $\mathbb{Z}_q^n \times \mathbb{Z}_q$ by choosing a vector $\mathbf{a}_i \in \mathbb{Z}_q^n$ uniformly at random, $e_i \in \mathbb{Z}_q$ according to $\chi$ and returning pairs of $(\mathbf{a}_i, \langle \mathbf{a}_i, \mathbf{s} \rangle + e_i \mod q) \in \mathbb{Z}_q^n \times \mathbb{Z}_q$.
\end{definition}

Additions are performed in $\mathbb{Z}_q$. We say that an algorithm solves LWE with modulus $q$ and error distribution $\chi$ if, for any $\mathbf{s} \in \mathbb{Z}_q^n$, given an arbitrary number of independent samples from $\mathcal{A}_{\mathbf{s}, \chi}$ it outputs $\mathbf{s}$ (with high probability). For $q=2$ corresponds to \textit{learning parity with noise} (LPN) problem. % TODO still needed with Search-LWE\ldots???

% TODO: error distribution => gaussian, paramter alpha, define?
% TODO matrix Schreibweise vs nicht matrix schreibweise
% TODO Define Search/Decision LWE? 
\begin{definition}[Search-LWE$_{n, q, m, \chi}$] % \cite{LP11}
  Search-LWE$_{n, q, m, \chi}$ asks for the recovery of the secret vector $\mathbf{s}$ given $m$ independent samples $(\mathbf{a}_i, z_i) \leftarrow \mathcal{A}_{\mathbf{s}, \chi}$ % TODO matrix
\end{definition}

\begin{definition}[Decision-LWE$_{n, q, m, \chi}$]
  Given $m$ samples, Search-LWE$_{n, q, m, \chi}$ asks to distinguish whether the samples were drawn from  $\mathcal{A}_{\mathbf{s}, \chi}$ or from a uniform distribution on $\mathbb{Z}_q^n \times \mathbb{Z}_q$.
\end{definition}

% TODO: show more about decoding problem?
% TODO: Einheitsmatrix in notation
% TODO: notation \mathbf{x} = (x_1, \ldots, x_n) is column vector and  \mathbf{x}^\intercal its corresponding tranpose
% TODO: add somewhere we assume that m > n
\subsubsection{LWE as a Decoding Problem} \label{sec:lwe-decoding}
We request $m$ samples $(\mathbf{a}_1, z_1), \ldots, (\mathbf{a}_m, z_m)$ where $z_i = \langle \mathbf{a}_i, \mathbf{s} \rangle + e_i \in \mathbb{Z}_q$. Let $A = \left[ \mathbf{a}_1 \cdots \mathbf{a}_m\right]$, $\mathbf{z} = \left[z_1, \ldots, z_m\right]^\intercal$ and $e = \left[e_1, \ldots, e_n\right]^\intercal$. Hence, we can reformulate LWE as a decoding problem as in \cite{GJS15}:
\begin{equation} \label{eq:lwe-decoding}
  \mathbf{z} =  \mathbf{A}^\intercal \mathbf{s} + \mathbf{e}
\end{equation} % TODO duplicate???
with generator matrix $\mathbf{A}$ for a linear code over $\mathbb{Z}_q$ and $\mathbf{z}$ as the received word. Finding the secret vector $\mathbf{z}$ is equivalent to finding the codeword $\mathbf{y} = \mathbf{A}^\intercal\mathbf{s}$ with minimum distance $\| \mathbf{y} - \mathbf{z} \|$. 

An LWE$_{n, q, m, \chi}$ instance with a secret vector $\mathbf{s}$ chosen according to a uniform distribution can be transformed into an LWE$_{n, q, m-n, \chi}$ instance with a secret vector $\hat{\mathbf{s}}$ chosen according to the error distribution $\chi$ at a loss of $n$ samples as follows: Let $\mathbf{A}_0 = \left[ \mathbf{a}_1 \cdots \mathbf{a}_n\right]$ where $\mathbf{a}_1, \ldots, \mathbf{a}_n$ are the first $n$ columns of $\mathbf{A}$. We introduce new variables $\hat{\mathbf{s}} = \mathbf{A}^\intercal_0 \mathbf{s}  - \left[z_1, \ldots, z_n\right]^\intercal = \left[e_0, \ldots, e_n\right]^\intercal$ and $\hat{\mathbf{A}} = \mathbf{A}_0^{-1} \mathbf{A} = \left[\mathbf{I} \; \hat{\mathbf{a}}_{n+1} \cdots \hat{\mathbf{a}}_{m}\right]$ and compute $\hat{\mathbf{z}} = \mathbf{z} -  \hat{\mathbf{A}}^\intercal \left[z_1, \ldots, z_n\right]^\intercal  = \left[\mathbf{0}, \hat{z}_{n+1} \cdots \hat{z}_{m} \right]^\intercal$. % TODO: where do we need that? maybe move there, cite somewhere \e.g. KF15, ACPS, BLPRS13 classical hardness of learning with errors

% TODO parameter choice: often prime q \in poly(n), \chi has mean zero and \sigma = \alpha \cdot q for some small \alpha, e.g. Regev: q \approx n^2, \alpha = 1/(\sqrt{2\pi n} \cdot \log_2^2 n)

\subsubsection{LWE as a BDD Problem} \label{sec:lwe-bdd} % \cite{LP11}
Solving LWE also corresponds to solving the \textit{Bounded Distance Decoding problem} (BDD) in the lattice $\Lambda(\mathbf{A}^\intercal) = \{ \mathbf{x} \in \mathbb{Z}_q^m \mid \exists \mathbf{s} \in \mathbb{Z}_q^n : \mathbf{x} = \mathbf{A}^\intercal \mathbf{s}  \mod q \}$, where the $m$ columns of $\mathbf{A}$ correspond to the vectors $\mathbf{a}_i \in \mathbb{Z}_q^n$ of $m$ independent LWE samples $(\mathbf{a}_i, z_i) \leftarrow \mathcal{A}_{\mathbf{s}, \chi}$ and the components $z_i$ correspond to a perturbed lattice point in $\Lambda(\mathbf{A}^\intercal)$. % TODO check ^\intercal



Best algorithm to solve LWE: Blum, Kalai, and Wasserman \cite{BKW03} with $2^{O(n)}$ samples and time. % TODO Sketch BKW?

Hardness: best algorithm exponential, extension of LPN (LPN believed to be hard), hard assuming worst-case hardness of \textsc{GapSVP} % TODO make sure that is correct
and \textsc{SIVP} \cite{Reg05, Pei09}. More details? Different cases for $q$ exponential/polynomial, approximation factors...
Hardness based on worst-case lattice problems => strong security guarantees, such as conjectured security against quantum computers...
 
Search to decision reduction => distiguishing is LWE samples from uniform samples sufficient, worst-case to average-case reduction => sufficient to solve distinguishing for uniform secret

\subsection{Short Integer Solution (SIS)}
The dual problem to LWE is the \textit{Short Integer Solution problem} (SIS).

- principle: given a set of set of uniformly random vectors $\mathbf{a}_1, \ldots, \mathbf{a}_m \in \mathbb{Z}_q^n$ find a subset of them or combination with small coefficients that sums to zero (modulo $q$). % TODO change or \cite{Reg10}

- introduced in \cite{MR04}, origins in \cite{Ajt96}, used for ´minicrypt´ primitives: one-way functions \cite{Ajt96}, collision resistant hash functions \cite{GGH96}, digital signature schemes \cite{GPV08, CHKP10}, and identification schemes \cite{MV03, Lyu08, KTX07} % TODO change or \cite{Reg10}

\begin{definition}[SIS Problem (Adapted from [\citealp{LS15}, Definition 3.1])]
  The problem \text{SIS}$_{n, q, m, \beta}$ is defined as follows: Given a uniformly random matrix $\mathbf{A}^{n\times m}$, find a vector $\mathbf{s} \in \mathbb{Z}_q^m$ such that $\mathbf{A} \cdot \mathbf{s} = 0 \; \text{mod } q$ and $0 < \| \mathbf{s}\| \leq \beta$.
\end{definition}

Finding such a vector corresponds to finding a short lattice vector in costets of the lattice $\Lambda^{\perp}(\mathbf{A}) = \{ y \mid \mathbf{A} \cdot y \mod q \}$ % TODO get source other than https://www.youtube.com/watch?v=LlPXfy6bKIY

Hardness: for any poly-bounded $m, \beta$ and for ``large enough'' prime $q$: SIS$_{n, q, m, \beta}$ is as hard as worst-case approx-SIVP (and \textsc{GapSVP}) to within $\beta \cdot \tilde{O}(\sqrt{n})$ factor 

% TODO: application, perhaps something as in file:///C:/Users/Nico/OneDrive/Studium/Informatik/6BA/Supplemental%20Material/[BC%20Micciancio]%20The%20Short%20Integer%20Solutions%20Problem%20and%20Cryptographic%20Applications.pdf would be nice


\subsection{Ring and Module Variants} 
- problem key sizes in LWE/SIS in $O(n^2)$ (matrix $\mathbf{A} \in \mathbb{Z}_q^{m \times n})$, where $m \in \Omega(n)$)% TODO change or \cite{Reg10}

- idea: introduce some sort of a structure in samples: $n$ power of two, $\mathbf{a}$ vectors in groups of size $n$, for each group $\mathbf{a}_1 = [a_1, \ldots, a_n]^\intercal$, $a_i$ are uniformly random in $\mathbb{Z}_q$, and $\mathbf{a}_i = [a_i, \ldots, a_n, -a_1, \ldots, -a_{i-1}]^\intercal$. Hence, $n$ vectors only need $O(n)$ memory, also speedups in operations by using FFT % TODO change or \cite{Reg10}

- formally: vectors are elements of the ring $\mathbb{Z}_q\left[x\right] / \left\langle x^n + 1 \right\rangle$ which we call $\mathcal{R}_q$ instead of the group $\mathbb{Z}_q^n$, $n$ power of two ensures that $x^n + 1$ is irreducible over the rationals% TODO change or \cite{Reg10}

- add more? % TODO

\begin{definition}[Ring-SIS Problem [\citealp{LS15}, Definition 3.3])]
  The problem \text{RSIS}$_{n, q, m, \beta}$ is defined as follows: Given $a_1, \ldots, a_n \in \mathcal{R}_q$ chosen independently from the uniform distribution, find $s_1, \ldots, s_n \in \mathcal{R}$ such that $\sum_{i=1}^m a_i \cdot s_i = 0 \mod q$ and $0 < \| \mathbf{s}\| \leq \beta$, where $\mathbf{s} = \left[s_1, \ldots, s_m\right]^\intercal \in \mathcal{R}^m$.
\end{definition} % TODO: need n? say sth more about n power of 2?
% TODO: matrix version?

\begin{definition}[Module-SIS Problem [\citealp{LS15}, Definition 3.3])]
  The problem \text{MSIS}$_{n, d, q, m, \beta}$ is defined as follows: Given $a_1, \ldots, a_n \in \mathcal{R}_q^d$ chosen independently from the uniform distribution, find $s_1, \ldots, s_n \in \mathcal{R}$ such that $\sum_{i=1}^m a_i \cdot s_i = 0\mod q$ and $0 < \| \mathbf{s}\| \leq \beta$, where $\mathbf{s} = \left[s_1, \ldots, s_m\right]^\intercal \in \mathcal{R}^m$.
\end{definition} % TODO: need n? say sth more about n power of 2?
% TODO: matrix version?

% TODO: LWE

While there exist special cases where the Ring structure of problem instances can be exploited in an attack on LWE or SIS, % TODO: find examples
in general, the hardness of Ring and Module variants is estimated by interpreting the coefficients of elements of $\mathcal{R}_q$ as vectors in $\mathbb{Z}_q^n$ \cite{ACDDPPVW18}.
We thus reduce Ring and Module instances as follows:
\begin{itemize}
  \item RLWE$_{n, q, m, \chi} \longrightarrow$ LWE$_{n, q, m \cdot n, \chi}$
  \item MLWE$_{n, d, q, m, \chi} \longrightarrow$ LWE$_{n \cdot d, q, m \cdot n, \chi}$
  \item RSIS$_{n, q, m, \beta} \longrightarrow$ SIS$_{n, q, m \cdot n, \beta}$
  \item MSIS$_{n, d, q, m, \beta} \longrightarrow$ SIS$_{n \cdot d, q, m \cdot n, \beta}$
\end{itemize}
Note that in the Ring and Module variants $n$ denotes the degree of the polnomial of the underlying Ring, %TODO: correct?
while in the standard variant, $n$ denotes the dimension of the secret. 
  
  







% TODO: replace "hardness" with "complexity of solving"?
\chapter{Algorithms and Estimates}

\section{Lattice Basis Reduction} % perhaps move to Algorithms
% For survey see Ngu11, NV10, MW16

% TODO: rewrite (taken from AGVW17)
Problem: usually ugly basis (long vectors...), we want a better basis with shorter and more orthogonal basis vectors...
- improve lattice basis quality => measure by hermite factor (compare shortest vector in basis to lattice volume) or approximation factor (compare shortest vector in basis to shortest lattice vector)
- algorithm finding vector with approximation factor $\gamma$ can be used to solve uSVP with gap $\lambda_2(\Lambda)/\lambda_1(\Lambda) > \gamma $
- best known theoretical bound by Slide reduction \cite{GN08a}, BKZ better in practice

% TODO possibly put Gram-Schmidt orthogonalization here? => Size reduction algorithm and HKZ reduction, see 2.4.1 in Chen13, not sure if needed
- measure quality of basis: Hermite factor  % TODO change or \cite{Reg10}

  * basis $\mathbf{B} = \left\{\mathbf{b}_1, \ldots, \mathbf{b}_m\right\}$, $m$-dimensional lattice $\Lambda(\mathbf{B})$ has root Hermite factor $\delta$ if
  \begin{equation} \label{eq:hermite}
    \| \mathbf{b}_1 \| \approx \delta^m \det(\Lambda)^{1/m}
  \end{equation}

  * use Geometric Series Assumption (GSA) \cite{Sch03} to obtain estimates for $\mathbf{b}_i$: % TODO: necessary? 
    \begin{equation} \label{eq:GSA}
      \| \tilde{\mathbf{b}}_i \| \approx \alpha^{i-1} \| \mathbf{b}_1 \|
    \end{equation}
    for $0 < \alpha < 1$
    \cref{eq:hermite} into \cref{eq:GSA} -> $\| \tilde{\mathbf{b}}_i \| \approx \alpha^{i-1} \delta^m \det(\Lambda)^{1/m}$
    with $\prod_{i-1}^m \| \tilde{\mathbf{b}}_i \| = \det(\Lambda)$ we get 
    \begin{align*}
      &\quad& \prod_{i-1}^m \| \tilde{\mathbf{b}}_i \| &\approx \prod_{i-1}^m \alpha^{i-1} \delta^m \det(\Lambda)^{1/m} \\
      \iff&\quad& \det(\Lambda) &\approx \delta^{2m} \det(\Lambda) \prod_{i-1}^m \alpha^{i-1}\\
      \iff&\quad& \delta^{-m^2}  &\approx \alpha^{\frac{m(m-1)}{2}}\\
      \iff&\quad& \delta^{-2}  &\approx \alpha^{(m-1)/m}\\
    \end{align*}
    Hence, $alpha \approx \delta^{-2}$ and 
    \begin{equation}
      \| \tilde{\mathbf{b}}_i \| \approx \delta^{-2(i-1) + m} \det(\Lambda)^{1/m}
    \end{equation}

  * good basis -> first Gram-Schmidt vectors become shorter (latter longer)

  * $\delta = 1.01$ feasible, $\delta = 1.007$ seems infeasible for now

  * gap between provable and experimental cost estimate to reach some hermite $\delta$ => provable results only give upper bounds, for practical security we need lower bound => combine theoretical results with experimental results

  * well-established estimate \cite{LP11}

In the following, we will focus on two related methods for lattice reduction. % A third approach, the Hermite, Korkine, Zolotarev (HKZ) reduction %TODO: write a sentence or two about it?

% TODO: algorithm LLL reduction, check out CWX13
\subsection{The LLL Algorithm}
The LLL algorithm was proposed by Lenstra, Lenstra and Lovász \cite{LLL82} and can be considered as a generalization of the two dimensional Lagrange reduction. The lagrange reduction reduces a basis of two basis vectors such that output basis satisfies $\|\mathbf{b}_1\| \leq \|\mathbf{b}_2\|$ and $\frac{|\left\langle\mathbf{b}_1, \mathbf{b}_2\right\rangle|}{\|\mathbf{b}_1\|} = |\mu_{2,1}| \leq \frac{1}{2}$). Intuitively, a multiple of the shorter vector $\mathbf{b}_1$ is subtracted from the longer vector $\mathbf{b}_2$ such that the resulting vector $\mathbf{b}_2'$ is as orthogonal to $\mathbf{b}_0$ as possible, i.e.  $\mathbf{b}_1' =  \mathbf{b}_1 - \lfloor\mu_{1,0}\rceil \mathbf{b}_0$. We set $\mathbf{b}_2 = \mathbf{b}_2'$ and repeat until nothing changes. 

A $\delta$-LLL reduced basis ensures two criterias \cite{LLL82}:
\begin{enumerate}
  \item Size reduced: $|\mu_{i,j}| \leq \frac{1}{2}$ for $1\leq i \leq n$ and $j < i$ \label{size-red}
  \item Lovász condition: $\delta \| \tilde{\mathbf{b}}_i \|^2 > \| \mu_{i+1, i} \tilde{\mathbf{b}}_i + \tilde{\mathbf{b}}_{i+1} \|^2$ for $1\leq i < n$
\end{enumerate}

Recall the definition of the Gram-Schmidt coefficients $\mu_{i, j} = \frac{\left\langle \tilde{\mathbf{b}}_j, \mathbf{b}_i\right\rangle}{\left\langle \tilde{\mathbf{b}}_j, \tilde{\mathbf{b}}_j\right\rangle}$. The LLL algorithm shown in \cref{alg:LLL} follows the notation in \cite{LLLReg04}. We start by computing the Gram-Schmidt orthogonalization of the input basis (\cref{alg:LLL-start}) and continue with a reduction step in which we update every basis vector $\mathbf{b}_i$ by pairwisely comparing and subtracting lower indexed basis vector just as in the Lagrange reduction (\cref{alg:LLL-red}) to ensure Criteria \ref{size-red}. Finally, vectors violating the Lovász condition are swapped (\cref{alg:LLL-swap}ff) and the process is repeated until nothing changes. The LLL algorithm can be used to find short vectors of at most $2^{n/2} \lambda_1(\Lambda)$ in polynomial time. Several floating-point variants have been suggested that can significantly speed up the runtime of LLL. For example, L$^2$ runs in $\mathcal{O}(n^2 \log^2 B)$, where $B$ is a bound on the norm of the input basis vectors \cite{NS05}. % TODO: newer versions?

\begin{algorithm2e}
\SetKwBlock{Begin}{function}{end function} % FROM https://cims.nyu.edu/~regev/teaching/lattices_fall_2004/ln/lll.pdf
\Begin($\delta\text{-LLL} {(}\mathbf{B} \in \mathbb{Z}^{m\times n} {)}$) % TODO nxn???
{
  Compute $\tilde{\mathbf{B}}$\label{alg:LLL-start}\\
  \For{$i=2, \dots, n$}{ 
    \For{$j=i-1, \dots, 1$}{
      $\mathbf{b}_i = \mathbf{b}_i - \lfloor\mu_{i, j}\rceil \mathbf{b}_j$\label{alg:LLL-red}\\
    }
  }
  \If{$\exists i \text{ \upshape such that } \delta \| \tilde{\mathbf{b}}_i \|^2 > \| \mu_{i+1, i} \tilde{\mathbf{b}}_i + \tilde{\mathbf{b}}_{i+1} \|^2$}{\label{alg:LLL-swap}
    tmp = $\mathbf{b}_i$\\
    $\mathbf{b}_{i} = \mathbf{b}_{i+1}$\\
    $\mathbf{b}_{i+1} = \mathbf{b}_{i}$\\
    Return $\delta$-LLL($\mathbf{B}$)\\
  }
  \Else{
    Return $\mathbf{B}$\\
  }
}
\caption{The $\delta$-LLL Algorithm \cite{LLL82}} \label{alg:LLL}
\end{algorithm2e}



\subsection{The BKZ Algorithm} \label{sec:BKZ}
% Rewrite, from Pla18

The Block Korkin-Zolotarev (BKZ) algorithm was proposed by Schnorr in 1987 and adapted by Schnorr and Euchner in \cite{SE91} and represents a family of lattice reduction algorithm. Essentially, BKZ iteratively divides the input basis into blocks of a lower dimension $k$ and calling an SVP oracle on each block. The output of the oracle is then used to obtain a basis of improved quality. 

\cref{alg:BKZ} presents the main concept of BKZ and follows the description in \cite{CN11} with some adjustments. Initially, we run an LLL reduction on the input basis $\left\{\mathbf{b}_1, \dots, \mathbf{b}_{n}\right\}$ and update the basis. In each $j$th iteration, we consider a block of $k$ basis vectors $\mathbf{b}_j, \dots, \mathbf{b}_{j+k-1}$. The vectors of the current block are projected onto the orthogonal complement of the span of vectors from previous iterations $\text{span}\left(\left\{\mathbf{b}_i \mid i \in [j-1]\right\}\right)$ (\cref{BKZ-span} - \ref{BKZ-proj}, we skip this step if the span is empty). Note that the orthogonal complement $A^\perp$  of a subspace $A$ is defined as the set of all vectors that are orthogonal to every vector in $A$. We then run an SVP oracle on the projected block to obtain a shortest vector $\mathbf{b}_\text{new}'$ in the projected lattice (\cref{BKZ-svp}) and reconstruct a lattice vector $\mathbf{b}_\text{new}$ of which $\mathbf{b}_\text{new}'$ is a projection \cref{BKZ-rec}. Note that in practice, the SVP oracle should include this step. If $\mathbf{b}_\text{new}$ is a new vector we insert it in our list of basis vectors before $\mathbf{b}_j$. Otherwise as nothing changed, we increment a counter $z$. Finally, we run LLL on all basis vectors up to index $j+i$ (including the possibly newly added vector). If no new lattice vectors can be found in $n$ iterations, the reduction terminates. After $n$ iterations, $j$ is reset to start over at the first block. The ouput of the algorithm is a BKZ$_k$-reduced basis. For $k=2$ we obtain an LLL-reduced basis in polynomial time and for $k=n$ an optimally HKZ-reduced basis in at least exponential time.%TODO needed? probably not, maybe just mention k=2 case, Pla18


\begin{algorithm2e} % TODO: check if block is not of dimension kxk
  \SetKwBlock{Begin}{function}{end function}
  \Begin($\text{BKZ} {(} \mathbf{B} = \{\mathbf{b}_1, \dots, \mathbf{b}_{n}\}, k \in [n]\backslash\{1\} {)}$) % TODO nxn???
  {
    z = 1; j = 0\\
    $\mathbf{B} = \text{LLL}(\mathbf{B})$ \\% LLL reduce and update basis
    \While{$z < n-1$}{ 
      $j = (j \mod (n-1)) + 1; l = \min(j + k -1, n); h = \min(l + 1, n)$\\
      $A = \text{span}\left(\left\{\mathbf{b}_i | i \in [j-1]\right\}\right)$\label{BKZ-span}\\
      \For{$i \in \{j, ..., l\}$}{
        \If{$A \neq \emptyset$}{
          $\mathbf{b}_i' = \pi_{A^\perp}(\mathbf{\mathbf{b}_i})$\label{BKZ-proj}\\
        }
        \Else{
          $\mathbf{b}_i' = \mathbf{b}_i$\\
        }
      }     
      $\mathbf{b}_\text{new}' = \text{SVP-Oracle}(\mathbf{b}_j', \dots, \mathbf{b}_{l}')$\label{BKZ-svp}\\
      Reconstruct $\mathbf{b}_\text{new} = \textstyle \sum_{i=j}^l\alpha_i \mathbf{b}_i$ with $\alpha_i \in \mathbb{Z}$ such that $\mathbf{b}_\text{new}' = \pi_{\left(\text{span}\left(\mathbf{b}_j, \dots, \mathbf{b}_{l}\right)\right)^\perp}(\mathbf{b}_\text{new})$\label{BKZ-rec}\\
      \If{$\mathbf{b}_\text{new}' \neq \tilde{\mathbf{b}}_j$}{
        $z=0;$ $\left\{\mathbf{b}_j, \dots, \mathbf{b}_{h} \right\}= \text{LLL}(\left\{\mathbf{b}_j, \dots, \mathbf{b}_{j-1}, \mathbf{b}_\text{new}, \mathbf{b}_j, \dots, \mathbf{b}_{h} \right\})$\\
      }
      \Else{
        $z = z + 1;$ $\left\{\mathbf{b}_j, \dots, \mathbf{b}_{h} \right\} = \text{LLL}(\left\{\mathbf{b}_j, \dots, \mathbf{b}_{h} \right\})$\\
      }
    }
  }
  \caption{The BKZ Algorithm \cite{SE91}} \label{alg:BKZ}
\end{algorithm2e} % TODO: check, maybe also with ABLR21, Alg 2 or 4

% TODO: explain how SVP oracle can be implemented => possibly outsource, section for cost models!!!

Several improvements have been suggested. The total number of rounds until termination is unknown and can be quite large. Hanrot \textit{et al.} \cite{HPS11a} show an early termination of BKZ still yields a very good output basis quality and propose $\frac{n^2}{k^2} \log n$ rounds as a bound. Th
Local preprocessing increases the quality of the current block basis by recursively calling BKZ with smaller block size. A variant known as progressive BKZ applies the recursion globally \cite{AWHT16}. If enumeration is used as an SVP oracle (see \cref{sec:costmodels}), extreme pruning can be applied to significantly reduce the search space. Gamma \textit{et al.} show that applying such a bounding function on the search tree reduces the running time by a much larger factor than the success probability. Repeating the search yields the desired speedup \cite{GNR10}. In addition, \cite{CN11} optimizes the enumeration radius by using experimental results. BKZ 2.0 incorporates a number of these techniques \cite{CN11}.

% BKZ runtime bounds: TODO rewrite, from MW16
% - runtime (without early termination) grows superpolynomially in $n$ \cite{GN08b}
% - best provable bounds on ouput after termination worse than that of slide reduction by at least polynomial factor (slide reduction [\cite{GN08a}] yields best theoretical results but in practice worse than BKZ)
% - calls to SVP oracle in arbitrary dimensions up to block size => runtime depends on worst-case Hermite constants of each block size 
% => simulation of BKZ needs to include estimation of runtime of SVP oracle for each block size
% TODO: use the following? 
% Versions:
% - slide reduction yields best theoretical result (2^n log log n /log n) \cite{GN08a} 
% - practical floating point version of LLL \cite{SE91} best practical results 

It is difficult to find hard runtime bounds for BKZ. The upper bound on the number of rounds is superexponential in the dimension $n$ for a fixed block size \cite{HPS11a, GN08b} before BKZ terminates by itself. In addition, calls to the SVP oracle is called in all dimensions $k' \leq k$ must be taken into account. 
Albrecht \textit{et al.} ignore these intricacies and estimate the cost of BKZ in clock cycles as $\rho \cdot n \cdot t_k$ where $\rho$ is the number of rounds needed and $t_k$ is the cost (in block cycles) of calling the SVP oracle on a block of dimension $k$ \citet{APS15}. The value $\rho$ is set to $8$ in the \textit{Estimator} derived from experiments in \cite{Chen13} that indicate that the most significant progress happens in the first $7-9$ rounds. 

% TODO: include? \cite{LP11} runtime according Linder-Peikert model, GSA: suggest $t_{\text{BKZ}}=1.8/\log_2{\delta} - 110$


% TODO perhaps look into: General Sieve Kernel implementation \cite{ADHKPS19} beats BKZ + pruned enum 



\subsection{Cost Models for Lattice Reduction} \label{sec:costmodels}
% TODO. write little introduction
In this section, we will look at various high level ideas to realize an SVP solver that can be used as a subroutine in BKZ and present up-to-date cost models from the literature. SVP is known to be NP-complete even for large constant approximation factors \cite{Ajt98, Khot05}. An exponential approximation factor can be achieved in polynomial time but is mostly insufficient for practical purposes \cite{LLL82}.  We will mainly focus on two classes of (nearly) exact SVP solvers, namely, enumeration algorithms and sieving algorithms. Enumeration algorithms can solve SVP in a lattice of dimension $k$ in $2^{\mathcal{O}(k \log k)}$ time and polynomial space. Sieving algorithms only need $2^{\mathcal{O}(k)}$ time, however, at the cost of exponential memory complexity. Only recently, progress in sieving strategies has given rise to BKZ implementations relying on sieving (e.g. the General Sieve Kernel (G6K) impelmentation \cite{ADHKPS19, DSW21}) that outperform enumeration implementations already in relatively small dimensions $\gtrsim 70$ in the classical setting \cite{ABLR21}. On the other hand, quantum speedups for enumeration are greater than for sieving. \citet{ANS18} show a quadratic cost reduction for enumeration, while the cost sieving only decreases by a factor of $2^0.027$ with idealized assumptions \cite{Laa15}. % assuming no depth restriction and quantum accessible RAM (qRAM)


% nice overview in https://cseweb.ucsd.edu/~daniele/LatticeLinks/Enum.html
\subsubsection{Enumeration} % see Chen13 for a better explanation
Enumeration aims to find the shortest vector by enumerating all lattice vectors within some bounded region. In general, we start with reducing the lattice basis to improve the basis quality. We then define a bound and iteratively project the lattice to the span of its Gram-Schmidt vectors beginning from $\tilde{\mathbf{b}}_n$ until we arrive at the lowest level of a one-dimensional subspace. We continue by enumerating all vectors of norm less than $r$ in the projected lattice and ``lift'' each of these vectors to the level above and repeat this process until we arive at the level from which we started. The search space can be thought of as a large tree of (projected) vectors on which we apply depth-first search. Note that the root of the tree here is at the lowest level and the leafs are the lattice vectors in our target lattice. %Consider a two-dimensional lattice with basis $\left\{\mathbf{b}_1, \mathbf{b}_2 \right\}$ and bound$r = \| \mathbf{b}_1 \|$. We project the lattice to the span of the Gram-Schmidt vector $\tilde{\mathbf{b}}_2$ and enumerate all the vectors within the  $r$... Not really needed
The low memory cost of enumeration is due to its similarities to depth-first search. 

A very early but very efficient variant was suggested by \citet{Kan83} with a worst-case runtime of $2^{\mathcal{O}(k \log k)}$. BKZ$_k$ using Kannan's enumeration algorithm as SVP oracle yields a short lattice vector of norm $\approx (k^{1/(2k)})^n\cdot \text{Vol}(\Lambda)^{1/n}$ \cite{HS07, ABFKSW20}. % TODO check if n or m
% TODO: write sth about root hermite factor, ... for the next sections or change next sections

In \cite{Chen13, ABFKSW20} we find a more concrete cost model of $\text{poly}(n) \cdot 2^{1/(2e) k \log k + - 0.995 k + 16.25}$ for BKZ 2.0 (see \cref{sec:BKZ}), where $\text{poly}(n)$ is the number of calls to the enumeration subroutine. BKZ 2.0 achieves a  root Hermite factor of $\left(\frac{k}{2\pi e} \cdot (\pi k)^{1/k}\right)^{\frac{1}{2(k-1)}}$. 

% TODO add to BKZ variants!!! => extended preprocessing and relaxing of search radius, 
% also add G6K (General Sieve Kernel) from ADHKPS19 (see more in sieving), running time of 2^Theta(k) but also 2^Theta(k) memory
The FastEnum algorithm in \citet{ABFKSW20} incorporates an idea called ``extended preprocessing'' and achieves a root Hermite factor of $k^{\frac{1}{2k}(1 + o(1))}$ in $\text{poly}(n) \cdot 2^{0.125k \log k - 0.050k + 56}$ time. The corresponding quantum algorithm reduces the runtime to $k^{k/16 + o(k)}$. In extended preprocessing, instead of preprocessing the current projected basis block of size $k$, the BKZ-reduction is applied to a block of higher dimension $\lceil (1+c)\cdot k\rceil$ for some constant $c$. Enumeration is faster on the first basis vectors as their Gram-Schmidt norms closely follow the Geometric Series Assumption \cite{MW16}. % TODO: move to BKZ ??? And perhaps switch BKZ with this section?

A tradeoff of runtime and success probability for ``relaxing'' the approximation and extreme pruning turns out to exponentially speed up the search \cite{LN20} and was combined with extended preprocessing in \cite{ABFKSW20} to further reduce the runtime of BKZ to $\text{poly}(n) \cdot 2^{0.125k \log k - 0.654k + 25.84}$ for a root Hermite factor of $k^(1/(2k))$. 

% TODO: what to do with poly(n)? just throw it away? 


% TODO: for table
% BKZ 2.0 & \cite{CN11, Chen13, ABFKSW20} & $2^{0.184k \log k - 0.995k + 16.25}$
% ABF+ Enum & \cite{ABFKSW20} & $2^{0.125k \log k}$ => root Hermite factor of $k^(1/(2k))$
% ABF+ Enum + O(1)& \cite{ABFKSW20} & $2^{0.125k \log k - 0.547k + 10.4}$
% ABF Q-Enum & \cite{ABFKSW20} & $2^{0.0625 k \log k}$ 
% ABLR Enum + O(1)& \cite{ABLR21} & $2^{0.125k \log k - 0.654k + 25.84}$



\subsubsection{Sieving} 
The second group of SVP solvers are sieving algorithms. In sieving, initially, we create a long list of randomly selected lattice points. The points in the list are then combined or ``reduced'' in some way to find points of smaller length. One way to achieve this is by finding a minimimal sublist of ``center'' points in the initial list such that spheres centered at these points cover all points list points. Subtracting the center points yields short lattice points. ListSieve \cite{MV10} uses a smaller initial list to divide the space into two half-spaces, one closer to the center and one closer to the respective point. The list is then used to reduce the length of newly sampled points as much as possible by subtracting each list vectors such that the result is located in the half-space closer to the center respectively. Once two points with a distance less than the target distance are found, they are subtracted and the result is returned. 
% TODO: nearest neighbor speedups: combining two vectors can only result in reduction only if their pairwise angle is less than $60°$. A candidate check of two vectors of similar length can thus be carried out by testing if the angle is at most $\pi/3$. \cite{BDGL16}

% nice overview and explanation in \cite{ADHKPS19}
\begin{table}[h]
  \centering
  \begin{tabular}{ll}
    \toprule
    Name & Cost Estimate \\\hline
    List Sieve \cite{MV10} & $2^{0.3199n + o(n)}$ time, $2^{0.1325 + o(n)}$ memory\\
    NV-sieve \cite{NV08, ADHKPS19} & $2^{0.415n + o(n)}$ time, $2^{0.2075n + o(n)}$ memory\\
    NV-sieve (quantum) \cite{NV08, ADHKPS19} & $2^{0.311n + o(n)}$ time, $2^{0.2075n + o(n)}$ memory\\
    Gauss sieve \cite{MV10, HK17} & $2^{0.415n + o(n)}$ time, $2^{0.2075n + o(n)}$ memory\\
    BGJ-sieve \cite{BGJ15} & $2^{0.311n + o(n)}$ time, $2^{0.2075n + o(n)}$ memory\\
    3-sieve \cite{BLS16, HK17} & $2^{0.3962n + o(n)}$ time, $2^{0.1887n + o(n)}$ memory\\
    BDGL-sieve \cite{BDGL16} & $2^{0.292n + o(n)}$ time\\
    BDGL-sieve (quantum) \cite{BDGL16} & $2^{0.265n + o(n)}$ time \\
    \bottomrule
  \end{tabular}
  \caption{Sieving Algorithms} % TODO: cite? from estimate all the?
  \label{tab:sieving}
\end{table}

\cref{tab:sieving} presents a list of currently best sieving algorithms. 
In the Nguyen-Vidick sieve \cite{NV08}, we iteratively reduce a pair of list points whose combined length is smaller than the longest list vector. The longest vector is then replaced by the result. The list length is fixed. 
In the Gauss sieve \cite{MV10}, we start with an empty list and a stack. In each step, a new point is either sampled or taken from the stack. We then attempt to reduce the new point with all points in the list. If a reduction is successful, the longer vector of the pair is replaced. If the longer vector was the list point, the replacement is inserted in the stack. If no reduction is possible, the stack points are moved back to the list. If the stack is empty, all list points are reduced pairwisely. In practice, the Gauss sieve outperforms Nguyen-Vidick sieve. 
The Becker-Gama-Joux sieve \cite{BGJ15} exploits coding theory to find vectors that are likely to be nearest neighbors. Similar vectors are stored in the same bucket to speed up the search for reduction candidates.
The 3-sieve \cite{BLS16, HK17} reduces the required list size by using triples instead of pairs of points for combination.
Finally, the Becker-Ducas-Gama-Laarhoven sieve \cite{BDGL16} applies locality sensitive hashing to create buckets of points in near neighborhood similar to the Becker-Gama-Joux sieve. % CHECK if that is ListDecoding

TODO: small section about use in tool... \cref{tab:costmodels} provides an overview of the cost models that are built into the Lattice Parameter Estimation.






% TODO: check out AKS01 algorithm 

% recent progress: % from DSW21
% - theoretical: Laa15, BGJ15, BDGL16, HKL1
%   techniques: 
%     Nearest Neighbor Search (Laa15)
%     Progressive Sieving (LM18)
%     Dimensions for Free (Duc18): lift many short vectors in sieving dimension to recover short vector in larger dimensions
% - practical performance: FBB+14, Duc18, LM18, ADHKPS19

% quantum: sieving uses near neighbor search as subroutine, which in turn uses black box search, Grover's quantum search algorithm reduces the complexity of black box search by square root \cite{Gro97}


% TODO also check out LM18, Duc18, ADHKPS19



%of length smaller than some bound $r$. Then, we choose some center points $\mathbf{v}'_i, i \in [l]$ for some $l \ll 2^{cn}$ of the list such that all points $\mathbf{v}_i$ of the initial list are covered by spheres of a radius $r' < r$ centered at these center points $\mathbf{v}'_i$. Finally, we compute short vectors by subtracting the centers. % TODO


% - CN11, APS15, ADPS16
% sieving \cite{BDGL16, Laa15,ADPS16, APS15,BDGL16}
% enumeration \cite{Kan87, MW15,FP85, CN11, APS15, HPSSWZ17}
% cost models (tabular overview) \cite{ACCD+19}
% - number of SVP calls \cite{ADPS16,Alb17}
% - upper bound on rounds is exponential \cite{HPS11a, GN08b}
% - lattice rule of thumbs: achievable root hermite factor $\delta_0 = k^{\frac{1}{2k}} $ % [Stehlé12 An introduction to lattice reduction] or 2^1/k [Stehlé13/16  An overview of lattice reduction algorithms] (oversimplification)





% primarily studied for SVP in euclidean norm
% two main types: 
% Classic Sieve \cite{AKS01}: create a long list of lattice vectors, then find shorter lattice vectors and discard longer
% List Sieve \cite{MV10}: start from empty list, find shorter vectors and append them to list

% best provable: $2^{n + o(n)}$ (sieving by averages)
% heuristic state of the art: $(3/2)^{n/2 + o(n)} \approx 2^{0.29n}$ 

% combinarial: create list of short random (possibly non-lattice) vectors, try to combine vectors in list to produce short lattice vectors
% provable versions not very useful (exponential space, random perturbations) but basis of heuristic variants used in practice 


% TODO: decide whether to take others
% \subsubsection{Others}
% - discrete Gaussian sampling % \cite{ADRS15}, adps?
% % TODO insert a table and reference somewhere else? Warum notwendig, wie kommt man darauf? ... alg laufen lassen, extrapolieren...
% - Micciancio-Voulgaris Algorithm - Voronoi-cell algorithm \cite{MV13} in $4^{n + o(n)}$ time and $2^{n+o(n)}$ space


% TODO: make chronology? 
% \begin{figure}
%   \begin{chronology}[5]{2000}{2021}{3ex}[\textwidth]
%     \event{2001}{AKS sieve $2^{\mathcal{O(n)}}$ time and space \cite{AKS01}}
%     \event{2010}{List sieve in $2^{3.2n}$ time and $2^{1.33n}$ space \cite{MV10}}
%     \event{2013}{Voronoi cell $2^{2n}$ time and $2^{n}$ space \cite{MV13}}
%     \event{2014}{Gaussian sampling $2^{n}$ time and space \cite{ADRS14}}
%   \end{chronology}
%   \caption{Provable Algorithms for SVP}
% \end{figure}

\begin{table}
  \centering
  \begin{tabular}{lll}
    \toprule
    Name & Reference & Cost model \\\hline
    Q-Core-Sieve & \cite{Laa15,ADPS16,AGPS20} & $2^{0.265n}$\\
    Q-Core-Sieve + O(1) & \cite{SAL+17} & $2^{0.265n + 16}$\\
    Q-Core-Sieve (min space) & \cite{SHRS17} & $2^{0.2975n}$\\
    Q-$\beta$-Sieve & \cite{NAB+17} & $n \cdot 2^{0.265n}$\\
    Q-$8d$-Sieve & \cite{BAA+17} & $8d \cdot 2^{0.265n + 16.4}$\\
    Core-Sieve & \cite{BDGL16,ADPS16,AGPS20} & $2^{0.292n}$\\
    Core-Sieve + O(1) & \cite{SAL+17} & $2^{0.292n + 16}$\\
    Core-Sieve (min space) & \cite{SHRS17} & $2^{0.368n}$\\
    $\beta$-Sieve & \cite{NAB+17} & $n \cdot 2^{0.292n}$\\
    $8d$-Sieve & \cite{DTGW17} & $8d \cdot 2^{0.292n + 16.4}$\\
    Q-Core-Enum + O(1)& \cite{SHRS17, Chen13,ACDDPPVW18} & $2^{\frac{0.18728n \log n - 1.0192 n + 16.1}{2}}$\\
    Lotus (Enum)& \cite{PHAM17, ACDDPPVW18} & $2^{0.125n \log n - 0.755 n + 2.254}$\\
    Core-Enum + O(1)& \cite{SHRS17, Chen13,ACDDPPVW18} & $2^{0.18728n \log n - 1.0192 n + 16.1}$\\
    $8d$-Enum (quadratic fit) + O(1) & \cite{BC0V17} & $8d \cdot 2^{0.000784 n^2 + 0.366 n + 0.875}$\\
    BKZ 2.0 Core-Enum & \cite{CN11, Chen13, ABFKSW20} & $2^{0.184k \log k - 0.995k + 16.25}$\\
    ABF+ Core-Enum & \cite{ABFKSW20} & $2^{0.125k \log k}$\\
    ABF+ Core-Enum + O(1)& \cite{ABFKSW20} & $2^{0.125k \log k - 0.547k + 10.4}$\\
    ABF Q-Core-Enum & \cite{ABFKSW20} & $2^{0.0625 k \log k}$ \\
    ABLR Core-Enum + O(1)& \cite{ABLR21} & $2^{0.125k \log k - 0.654k + 25.84}$ \\
    \bottomrule
  \end{tabular}
  \caption{Cost Models Overview} % TODO: cite? from estimate all the?
  \label{tab:costmodels}
\end{table} % TODO: add to tool, maybe change 8d thing to an optional parameter!!! => reduces number of lists, throw out silly ones...
% TODO reference in text
% TODO: add description of "core"


\section{Algorithms for Solving LWE}
\subsection{Overview}
% TODO from LP11, change
% TODO: maybe include graphic of three approaches
Distinguishing attacks (MR09, RS10): distinguish (with noticeable advantage) LWE instance from uniformly random => break semantic security of LWE-based cryptosystem with same advantage (typically), find short nonzero integral vector $\mathbf{v}$ s.t. $\mathbf{A}^\intercal \mathbf{v} = 0 \mod q$ => short vector in (scaled) dual of LWE lattice $\Lambda(\mathbf{A})$ % for A mxn matrix else switch t
then test whether $\langle \mathbf{v}, z \rangle$ is close to zero mod q. If uniform test accepts with prob 1/2, if LWE with parameter s, $\langle \mathbf{v}, \mathbf{z} \rangle = \langle \mathbf{v}, \mathbf{e} \rangle \mod q$, Gaussian mod q with parameter $\| \mathbf{v} \| \cdot s$. If that's not much larger than q, advantage for distinguishing very close to $\exp(-\pi (\| \mathbf{v} \| s/q)^2)$. high confidence needs $\| \mathbf{v} \| \leq q/(2s)$ 
advantage an computational effort need to be balanced (often inverse distinguising advantage is in total cost of attack)

\subsubsection{Dual Attacks}
reduce LWE to SIS
recover secret vector by finding a short vector in the dual lattice $\Lambda(\mathbf{A}^\intercal)^{\perp} = \{ \mathbf{y} \in \mathbb{Z}^m \mid \mathbf{A} \mathbf{y} = \mathbf{0} \mod q\}$ generated by the rows of $\mathbf{A}$ and scaled by q.
\subsubsection{Primal Attacks}

lattice reduction algorithms solve SIS and BDD 


\subsubsection{Direct} % or combinatorial

% TODO: based on GSJ15, rephrase
- algebraic approach Arora and Ge with subexponential complexity when $\sigma \leq \sqrt{n}$, else fully exponential, mainly of asymptotic interest (higher complexity than others)


- combinatorial algorithms: BKW as basis \cite{BKW03}, resembles generalized birthday approach by Wagner, % Wagner, D.: A generalized birthday problem. In: Advances in cryptology CRYPTO 2002, pp. 288304. Springer (2002)
originally for solving LPN, can be analyzed => explicit complexity for different LWE instances, theoretical analysis and actual performance close,
very memory expensive (often same order as time complexity) 



\subsection{BKW \cite{BKW03}}
The Blum, Kalai and Wasserman (BKW) algorithm was originially designed to solve the Learning Parity with Noise problem (LPN) \cite{BKW03}. In \cref{sec:lwe} we pointed out that LPN is a subproblem of LWE and Albrecht \textit{et al.} adapted BKW to LWE in \cite{ACFFP15a}. The runtime and memory complexity of BKW is in $2^{\mathcal{O}(n)}$ for an LWE instance with secret dimension $n$ prime modulus $q \in \text{poly}(n)$. The number of samples $m$ must be sufficiently large (in $\mathcal{O}(n \log n)$). % TODO check number of samples, APS15 say they need unrestricted access to LWE oracle
% TODO: 

BKW falls into the regime of dual attacks, that is, it solves LWE by finding a short vector $\mathbf{s}$ in the scaled dual lattice $\Lambda(\mathbf{A}^\intercal)^{\perp}$. 

% TODO: not sure if lattice vectors are only in Z_q^m (according to APS15), TODO: how exactly is BKW in the dual lattice? intuition missing... if it uses SIS strategy, maybe consider using it for solving SIS? problem: number of samples?

three stages \cite{ACFFP15a}: sample reduction, hypothesis testing and back substitution

\paragraph{Sample Reduction.} In the following, we present an outline of the main BKW algorithm. The steps in \cref{alg:BKW} are inspired by the textual description in \cite{GJS15} with minor adjustments in notation. 

% TODO: preprocessing: need secret distribution that follows error distribution? \cref{sec:lwe-decoding}


\begin{algorithm2e}
\SetKwBlock{Begin}{function}{end function}
\Begin($\text{BKW} {(}\mathbf{A} \in \mathbb{Z}^{n\times m},\mathbf{z} \in \mathbb{Z}^m, b \in \mathbb{Z}, d\in \mathbb{Z}{)}$)
{
  $i = 1$\\
  $\mathbf{A}^{(i)} = \mathbf{A}$\\
  $\mathbf{z}^{(i)} = \mathbf{z}$\\
  \While{the last $n-d$ coefficients of the columns of $\mathbf{A}^{(i)}$ are nonzero}{ % TODO
    // BKW step\\
    $j = 1$\\
    $\mathbf{T}^{(i)} = []$ \Comment Collision table\\
    \For{$k = 1, \ldots, m^{(i)}$}{
      // $m^{(i)}$ is number of columns in $\mathbf{A}^{(i)}$\\
      \uIf{last $(i\cdot b)$ coefficients of $\mathbf{a}_k^{(i)}$ are zero}{
        $\mathbf{a}_j^{(i+1)} = \mathbf{a}_k^{(i)}$\\
        $z_j^{(i+1)} = z_k$\\
        $j = j + 1$\\
      }
      \uElseIf{no match for $\mathbf{a}_k^{(i)}$ in $\mathbf{T}$}{
        $\mathbf{T} = \mathbf{T} + \left[\mathbf{a}_k^{(i)}\right]$ \Comment append to collision set
      }
      \uElseIf{match $\mathbf{a}_l^{(i)}$ for $\mathbf{a}_k^{(i)}$ is found}{
        \uIf{$\mathbf{a}_l^{(i)}$ matches $\mathbf{a}_k^{(i)}$ in the last $(i\cdot b)$ components}{
            $\mathbf{a}_j^{(i+1)} = \mathbf{a}_k^{(i)} - \mathbf{a}_l^{(i)}$; \Comment last $i \cdot b$ coefficients of $\mathbf{a}_j^{(i+1)}$ are now zero\\
            $z_j^{(i+1)} = z_k^{(i)} - z_l^{(i)} = y_j^{(i)} + e_j^{(i)}$, where $y_j^{(i)} = \left\langle \mathbf{s}, \mathbf{a}_j^{(i)}\right\rangle$ and $e_j^{(i)} = e_k^{(i)} - e_l^{(i)}$ \label{alg:BKW-z1}\\ 
            $j = j + 1$\\
        }
        \uElseIf{the negation of $\mathbf{a}_l^{(i)}$ in $\mathbb{Z}_q^n$ matches $\mathbf{a}_k^{(i)}$ in the last $(i\cdot b)$ components}{
          $\mathbf{a}_j^{(i+1)} = \mathbf{a}_k^{(i)} + \mathbf{a}_l^{(i)}$\\
          $z_j^{(i+1)} = z_k^{(i)} + z_l^{(i)} = y_j^{(i)} + e_j^{(i)}$, where $y_j^{(i)} = \left\langle \mathbf{s}, \mathbf{a}_j^{(i)}\right\rangle$ and $e_j^{(i)} = e_k^{(i)} + e_l^{(i)}$\label{alg:BKW-z2}\\
          $j = j + 1$\\
        } % TODO: maybe put both cases into one step?
      }
    }
    $i = i + 1$\\
    // Calculate input for next BKW step\\
    $\mathbf{A}^{(i)} = (\mathbf{a}_1^{(i)} \cdots \mathbf{a}_{j-1}^{(i)})$\\
    $\mathbf{z}^{(i)} = (z_1^{(i)}, \ldots, z_{j-1}^{(i)})$\\
  }
  Return $(\mathbf{A}^{(i)}, \mathbf{z}^{(i)})$
}
\caption{BKW (Sample Reduction)}\label{alg:BKW}
\end{algorithm2e} % TODO check

For the algorithm, we use the matrix notation of LWE as in \cref{eq:lwe-decoding}, i.e. $\mathbf{z} = \mathbf{A}^\intercal \mathbf{s} + \mathbf{e}$. BKW consists of a series of BKW steps that iteratively reduce the dimension of input matrix $\mathbf{A}$ by finding collisions of its column vectors in the currently examined block of $b$ entries. We start from the last $b$ entries of $\mathbf{A}^{(1)} = \mathbf{A}$. In every step $i$, we maintain a collision table $\mathbf{T}^{(i)}$ and loop over the columns $\mathbf{a}_k^{(i)}$ of $\mathbf{A}^{(i)}$ and distinguish between the following cases: (1) If $\mathbf{a}_k^{(i)}$ only has zero entries in the examined block, pass $\mathbf{a}_k^{(i)}$ and $z_k^{(i)}$ to the next step, (2) if no match of $\mathbf{a}_k^{(i)}$ or the negation of $\mathbf{a}_k^{(i)}$ can be found in the collision table, add $\mathbf{a}_k^{(i)}$ to the colission table, and (3) if a match $\mathbf{a}_l^{(i)}$ is found, compute $\mathbf{a}_l^{(i)} + \mathbf{a}_k^{(i)}$ or in the case of a negation match $\mathbf{a}_l^{(i)} - \mathbf{a}_k^{(i)}$ (in $\mathbb{Z}_q$) such that the last $b$ nonzero entries cancel out. By exploiting the symmetry of $\mathbb{Z}_q$ in this way, in every step we obtain at most $(q^b - 1)/2$ columns with distinct coefficients in the current $b$ entries. We also make note of ``observed symbols'' $z_j^{(i)}$ that represent the combination of two samples given their respective matching columns (see lines \ref{alg:BKW-z1}, \ref{alg:BKW-z2} for more details). 

In each BKW step, the number of columns (and samples) decreases by at least $(q^b - 1)/2$ (size of the colusion set) and the variance of the error distribution $\sigma^2$ increases by a factor of two. The algorithm terminates after $t = \lceil b / (n - d)\rceil$ steps returns a a set of observed symbols $\mathbf{z}^{(t)}$ and a corresponding reduced matrix $\mathbf{A}^{(t)}$ in which only the first $d$ rows have nonzero entries. Parameter $d$ should be set to $1$, as in the original BKW algorithm, or $2$ for the best performance \cite{ACFFP15a}.

The remaining part $\mathbf{s}'$ of the secret vector $\mathbf{s}$ is then guessed by means of hypothesis testing. After $t$ steps the error term $\left(\mathbf{z}_j^{(t)} - \left\langle \mathbf{s}', \mathbf{a}_j^{(t)}\right\rangle\right)$ with $j \in [m']$ of the $m'$ remaining observed symbols follows a Gaussian distribution $\chi$ with noise $\sigma'^2 = 2^t\cdot \sigma^2$ (see Lemma 1 in \cite{ACFFP15a}). % TODO: Lemma: Let $X_0, \dots, X_{m-1}$ be independent random variables with $X_i \tilde \mathcal{N}(\mu, \sigma^2)$. Then their sum $X = \sum_{i=0}^{m-1} X_i$ is distributed according to $\mathcal{N}(m\mu, m\sigma^2)$.
We can test the noise of the error term for all $\mathbf{s}'' \in \mathbb{Z}_q^d$ against the hypothesized noise  $\sigma'^2$ by means of the log-likelihood ratio (for details we again refer to \cite{ACFFP15a}) and are thus able to determine $\mathbf{s}'$ given sufficiently many samples $m'$. 

Finally, we can apply back substitution to recover all elements of $\mathbf{s}$. We again apply a similar procedure as in \cref{alg:BKW} to reduce a number of columns from the colission tables computed in the Sample Reduction step and obtain $m'$ columns with $d+d'$ nonzero entries and their corresponding ``observed symbols''. We then substitute the part of $s$ that was recovered in the previous steps and recover the next part of $\mathbf{s}$ by hypothesis testing and repeat the process until we have found $\mathbf{s}$. 

% TODO: runtime complexity change to APS15 Theorem 3, 4 (DTV15) or Cor 4, or try to take complexity from GSJ15
\begin{theorem}[BKZ Complexity \cite{ACFFP15a}, Corollary 2]
  Let $(\mathbf{a}_i, z_i)$ be samples following $\mathcal{A}_{\mathbf{s}, \chi}$, set $a = \lfloor \log_2(1/(2\alpha)^2)\rceil$, $b = n/a$ and $q$ a prime. Let $d$ be a small constant $0 < d < \log_2(n)$. Assume $\alpha$ is such that $q^b = q^{n/a} = q^{n/\lfloor \log_2(1/(2\alpha)^2)\rceil}$ is superpolynomial in $n$. Then, given these parameters the cost of the BKW algorithm to solve Search-LWE is 
  \begin{equation}\label{eq:BKZ-complexity}
    \left(\frac{q^b-1}{2}\right) \cdot \left(\frac{a(a-1)}{2} \cdot (n + 1) \right) + \left\lceil(\frac{q^b}{2}\right\rceil \cdot \left(\left\lceil(\frac{n}{d}\right\rceil + 1\right) \cdot d \cdot a + \text{poly}(n) \approx (a^2 n) \cdot \frac{q^b}{2}
  \end{equation}
  operations in $\mathbb{Z}_q$. Furthermore,
  \begin{equation}
    a \cdot \left\lceil(\frac{q^b}{2}\right\rceil + \text{poly}(n)\textit{ calls to } \mathcal{A}_{\mathbf{s}, \chi}\textit{ and storage of }  \left(a \cdot \left\lceil(\frac{q^b}{2}\right\rceil \cdot n\right) \textit{ elements in } \mathbb{Z}_q \textit{ are needed.}
  \end{equation}
\end{theorem}

The first summand of \cref{eq:BKZ-complexity} roughly corresponds to the cost of creating the colission tables and the second summand is the cost of backsubstitution. For a more detailed cost analysis, see Theorem 2 in \cite{ACFFP15a}.


\subsubsection*{Coded-BKW \cite{GJS15}}
- change BKW step -> more column entries are removed, but additional noise
- index set $I$, $\mathbf{x}_I$ is part of $\mathbf{x}$ with entries indexed by $I$
- step $i$: $I$ set of $b$ positions to be removed, fix some $q$-ary linear $\left[N_i, b\right]$ code $\mathcal{C}_i$ with $q^b$ codewords, find the closest codeword $\mathbf{c}_I \in \mathcal{C}$ for every input vector $\mathbf{a}_I$ such that $\mathbf{a}_I = \mathbf{c}_I + \mathbf{e}_I$, where the error part $\mathbf{e}_I \in \mathbb{Z}_q^{N_i}$ is minimized by a decoding procedure.

Finally, we subtract two vectors and their correpsonding samples and pass the result to the next BKW step. Consider the inner product$\left\langle \mathbf{s}_{I}, \mathbf{a}_{I} \right\rangle = \left\langle \mathbf{s}_{I}, \mathbf{c}_{I} \right\rangle + \left\langle \mathbf{s}_{I}, \mathbf{e}_{I} \right\rangle$. In the subtraction, only the error part $\left\langle \mathbf{s}_{I}, \mathbf{e}_{I} \right\rangle$ remains. 

% TODO
% TODO: cost estimate??? too complex to include here...




% - Various improvements have been suggested since. For example, \cite{AFFP14} and \cite{KF15} use modulus switching for binary-LWE and other small secret variants \cite{AFFP14} and multidimensional Furier transformations can be used to speed up operations \cite{DTV15}. % TODO define binary LWE \cite{AFFP14} and find others, what is modulus switching, how do they apply FFT?
% lazy modulus switching \cite{KF15}: modulus is switched from $q$ to some $p<q$ and each coefficient $x$ of vectors to be compared is set to $xp/q \in $\mathbf{Z}_p$. We obtain an additional error term, but the number of vectors needed in a block is reduced from $q^b$ to $p^b$. The modulus switching can be applied in each step with decreasing $p$ as the error term due to the standard BKW cancel operations increases and thus allows for a larger additional error term.

% modified BKW step -> coded-BKW step to cancel out more positions in the $\mathbf{a}$ vectors than traditional BKW step

% map part of $\mathbf{a}$ vector into nearest codeword in lattice code (linear code over $\mathbb{Z}_q$, Euclidean distance)

% introduces some noise, can be kept small by appropriate parameters

% pair of $\mathbf{a}$ vectors map to same codeword => add together to create new sample with part of $\mathbf{a}$ vector cancelled

% samples are input to next step in BKW procedure

% additional steps using discrete FFT

% slightly modified for BINARY-LWE (secret vector uniformly chosen from $\{0, 1\}^n$) greatly increases performance




\subsection{Dual Attack \cite{MR09}}
"Gama and Nguyen \cite{GN08b}: (in)feasibility of obtaining various Hermite factors
natural distinguishing attack on LWE by finding one relatively short vector in associated lattice"
% TODO: check out 6.4 in ADPS16
% TODO: main description in SIS section




\subsection{Decoding Attack \cite{LP11}} \label{sec:decoding}

combines lattice basis reduction followed by an enumeration algorithm (bounded-distance decoding with preprocessing?) => time/success tradeoff
specifically for LWE, exploits structural properties of LWE
on search version of LWE problem, approach preferable to distinguishing attack on decision LWE in \cite{MR09, RS10}, same or better advantage than distinguishing attack using lattice vectors of lower quality => runtime is smaller
post-reduction: simple extension of Babai's ``nearest-plane'' algorithm \cite{Bab85} % TODO describe
=> trade basis quality against decoding time
related to Klein's (de)randomized algorithm \cite{Kle00} for bounded-distance decoding

use entire reduced basis, post-reduction part is fully parallelizable
% prerequisites: babai's nearest plane algorithm (at least describe textually?), fundamental parallelipiped P_1/2


% TODO: algorithm Babai's NearestPlane % FROM https://cims.nyu.edu/~regev/teaching/lattices_fall_2004/ln/lll.pdf
LLL reduction to input Lattice, integer combination of basis vectors close to target (like inner loop in reduction step of LLL), seek vector in lattice close to target, finds output that is in fundamental parallelipiped $\mathcal{P}(\mathbf{B})$ \cref{eq:fundamental-parallelipiped} => if error vector not in $\mathcal{P}(\mathbf{B})$, secret is not restored % TODO
=> basis quality has to be sufficiently good
\begin{algorithm2e}
\SetKwBlock{Begin}{function}{end function}
\Begin($\text{NearestPlane} {(}\mathbf{B} \in \mathbb{R}^{m \times n},\mathbf{t}\in \mathbb{R}^{m}{)}$)
{
  run $\delta$-LLL on basis $\mathbf{B}$ with $\delta=\frac{3}{4}$\\ % TODO put outside of algorithm
  $\mathbf{b} = \mathbf{t}$\\
  \For{$i = n, \dots, 1$}{
    $c_i = \text{round}(\langle \mathbf{b}, \tilde{\mathbf{b}}_i\rangle /  \langle \tilde{\mathbf{b}}_i, \tilde{\mathbf{b}}_i\rangle)$
    $\mathbf{b} = \mathbf{b} - c_i \mathbf{b}_i$ 
  }
  output $\mathbf{t} - \mathbf{b}$
}
\caption{Babai's Nearest Plane Algorithm \cite{Bab85}}\label{alg:babai} % TODO change algorithm to match below or just take intuitive description
\end{algorithm2e}

Output is a lattice vector $\mathbf{v} \in \Lambda(\mathbf{B})$ such that $\|\mathbf{v} - \mathbf{t}\| \leq 2^{n/2} \text{dist}(\mathbf{t}, \Lambda(\mathbf{B}))$ % define distance to lattice, change 

Goal: recover lattice vector relatively close to target vector
Intuition:
 - project $\mathbf{t}$ to $\text{span}(\mathbf{B})$
 - from $i=n, \dots, 1$ find closest hyperplane $c_i \tilde{\mathbf{b}}_i + \text{span}(\mathbf{b}_1, \dots, \mathbf{b}_i)$ to the projection, subtract $c_i \mathbf{b}_i$ from the projection and continue % a little more detail
 - output vector is $\sum_{i=1}^n c_i \mathbf{b}_i$
% TODO: better description \as in BBDFGM14
for every basis vector $\mathbf{b}_i$ find $c_i$ such that distance between target and hyperplane spanned by $\mathbf{b}_1, ..., \mathbf{b}_{i-1}$ and shifted by $c_i \tilde{\mathbf{b}}_i$  is minimal % by using Gram-Schmidt vector
, subtract $c_i \mathbf{b}_i$ from target vector and continue for $i=n, \dots, 1$. After the last iteration $\sum_{i=1}^n  c_i \mathbf{b}_i$ is returned. 

Application to LWE: $\mathbf{t} = \mathbf{A}^\intercal\mathbf{s}+\mathbf{e}$ => we get $\mathbf{v}$ where $\mathbf{t}- \mathbf{v} = \mathbf{e}$ is in fundamental parallelipiped of Gram-Schmidt basis


Generalized version by \cite{LP11}:
Problem: in reduced basis last Gram-Schmidt vectors of B short, first long => long and skinny parallelipiped, Gaussian e unlikely to be in it => incorrect answer from NearestPlane

=> generalized version admitting time/success tradeoff
recurse on some $d_i \geq 1$ distinct planes in ith 

\begin{algorithm2e}
\SetKwBlock{Begin}{function}{end function}
  \Begin($\text{GeneralizedNearestPlane} {(} \mathbf{B} \in \mathbb{R}^{m \times k},\mathbf{t} \in \mathbb{R}^{m}, \mathbf{d} \in {(}\mathbb{Z}^+{)}^k {)}$) 
  { % TODO make sure that basis is always R^n, not Z^n
    \If{k = 0}{
      Return $\mathbf{0}$\\
    }
    \Else{
      Compute projection $\mathbf{v}$ of $\mathbf{t}$ onto $\text{span}(\mathbf{B})$\\
      Compute the $d_k$ distinct integers $c_1, \dots, c_{d_k}$ closest to $\langle \mathbf{v}, \tilde{\mathbf{b}}_k\rangle /  \langle \tilde{\mathbf{b}}_k, \tilde{\mathbf{b}}_k\rangle)$\\
      Return $\bigcup_{i \in \{1, \dots, d_k\}} (c_i \cdot \mathbf{b}_k + \text{GeneralizedNearestPlane}(\{\mathbf{b}_1, \dots, \mathbf{b}_{k-1}\}, (d_1, \dots, d_{k-1}), \mathbf{v} - c_i \cdot \mathbf{b}_k))$\\
    }
  }
\caption{Generalized Nearest Plane Algorithm \cite{LP11}}\label{alg:GeneralizedNearestPlane}
\end{algorithm2e}
Instead of choosing only the nearest plane in each iteration step, \cref{alg:GeneralizedNearestPlane} selects a variable amount $d_k$ of distinct planes in each step. As a consequence, the fundamental parallelipiped of the Gram-Schmidt basis is stretched in the direction of $\tilde{\mathbf{b}}_k$. The values of $\mathbf{d}$ should be chosen such that the covered area is approximately the same in each direction (i.e. by maximizing $\min_i(d_i \cdot \|\tilde{\mathbf{b}}_i\|)$). In particular this implies that the $d_k$ are larger for larger $k$ as the Gram-Schmidt vectors have a smaller length. % TODO combine with section above 
Compared to \cref{alg:babai} the runtime increases by a factor $\prod_{i \in \{1, \dots, d_k\}} d_i$, however, the recursion step can be fully parallelized.

It should be evident that a lower quality of the reduced input basis can be compensated for by increasing the values of $\mathbf{d}$. Hence we can adjust the input parameters for the lattice reduction and \cref{alg:GeneralizedNearestPlane} to minimize the runtime given a fixed required success probability. % TODO perhaps reformulate
% TODO: add exact success probability? 




\subsection{Primal-uSVP \cite{ADPS16, BG14}} % => AFG14, Kan87
% TODO: taken from ADPS16, change
BKZ: reduce lattice basis using SVP oracle in smaller dimension $b$, known that number of calls to oracle polynomial
- enumeration algorithm as oracle: in super-exponential time
- sieve algorithms as oracle: exponential time but so far slower in practice for accesible dimensions $b\approx 130$

primal attack: construct unique-SVP instance from LWE instance % TODO check if u-SVP is defined
LWE instance $(\mathbf{A}, \mathbf{z} = \mathbf{A}^\intercal \mathbf{s} + \mathbf{e})$
construct lattice 
\begin{equation}
  \Lambda = \left\{ \mathbf{x} \in \mathbb{Z}^{m+n+1} \mid (\mathbf{A}^\intercal | -\mathbf{I}_m | -\mathbf{b})\mathbf{x} = \mathbf{0} \mod q \right\} 
\end{equation}
lattice has dimension $d=m+n+1$, volume $q^m$ % TODO show
and unique-SVP solution $\mathbf{v} = (\mathbf{s}, \mathbf{e}, 1)$ % TODO show

success condition:
- geometric series assumption known to be optimistic from attacker's point of view
=> finds basis with Gram-Schmidt norms $\|\tilde{mathbf{b}}_i\| = \delta^{d - 2i-1} \cdot \text{Vol}(\Lambda)^{1/d}$ and $\delta = ((\pi b)^{1/b} \cdot b/2\pi e)^{1/2(b-1)}$ % see APS15 and Chen13 thesis
unique short vector $\mathbf{v}$ is detected if projection of $\mathbf{v}$ onto span of last $b$ Gram-Schmidt vectors is shorter than $\tilde{mathbf{b}}_{d-b}$, norm of projection is expected to be $\gamma \sqrt{b}$ => attack successful iff $\gamma \sqrt{b} \leq \delta^{d - 2i-1} \cdot q^{m/d}$
% TODO define span, e.g. vector space spanned by a number of vectors
% TODO check if einheitsmatrix is defined

LWE as inhomogeneous-SIS (ISIS)

% FROM APS15:
As in \cref{sec:decoding}, we view the LWE$_{n, q, m, \chi}$ instance $(\mathbf{A}, \mathbf{z})$ as a BDD instance in the  $q$-ary lattice $\Lambda(\mathbf{A}^\intercal) = \{ \mathbf{y} \mid \exists \mathbf{x} \in \mathbb{Z}_q^n : \mathbf{y} = \mathbf{A}^\intercal \mathbf{x}  \mod q \}$ \cref{sec:lwe-bdd} generated by rows of LWE instance. The target vector is $\mathbf{z}$. % TODO add to lwe as BDD/combine

Recall the $\gamma$-uSVP problem. Given a lattice $\Lambda$ where $\lambda_2(\Lambda) > \gamma \lambda_1(\Lambda)$, we are asked to find shortest nonzero vector in $\Lambda$. 
In the primal attack, instead of directly solving BDD, we reduce BDD to uSVP, i.e., we reduce a BDD instance to a $\gamma$-uSVP instance. By solving $\gamma$-uSVP, we obtain a solution to BDD. 
To do this we apply Kannan's embedding technique \cite{Kan87}. % TODO: quote theorem? For any $\gamma \geq 1, there is a polynomial time Cook-reduction from BDD$_{1/(2\gamma)}$ to $\gamma$-uSVP.
Intuitively, Kannan's embedding creates a lattice with uSVP structure. We know that $\mathbf{A}^\intercal \mathbf{s}\mod q$ is the closest vector to the target $\mathbf{z} =\mathbf{A}^\intercal \mathbf{s} + \mathbf{e}^\intercal \mod q$ in $\Lambda(\mathbf{A}^\intercal)$. We now add a linearly indpenendent basis vector  $(\mathbf{z}, \mu)$ and append a zero coefficient to each basis vector of the original lattice (i.e. the rows of $\mathbf{A}$). Thereby, we ensure that the new lattice contains the vector $[-\mathbf{e}, -\mu]^\intercal$ as $[\mathbf{A} \mid \mathbf{0}]^\intercal \mathbf{s} - 1 \cdot [\mathbf{z}^\intercal, \mu] = [-\mathbf{e}, -\mu]^\intercal$. 

More formally, let $\mathbf{B}$ be a basis of $\Lambda(\mathbf{A}^\intercal)$ % TODO: how? is \Lambda(\mathbf{A}^\intercal) not q-ary? => is A^\intercal not a basis?
and an embedding factor $\mu = \text{dist}(\mathbf{z}, \Lambda(\mathbf{A}^\intercal)) = \| \mathbf{z} - \mathbf{s}\|$ where $\mathbf{s}$ is the secret vector of the LWE instance. A relatively close approximation of $\mu$ can be guessed in polynomial time (see \cite{LM09} for more details). We now embed $\Lambda(\mathbf{A}^\intercal)$ into $\Lambda(\mathbf{B}')$ with $\gamma$-uSVP structure as follows: % TODO: check if dist = \| \mathbf{z} - \mathbf{s}\| is correct, perhaps just a vector v such that equation is fulfilled???
\begin{equation}
  \mathbf{B}' = \begin{pmatrix}
    \mathbf{B} & \mathbf{z}\\
    \mathbf{0}^\intercal & \mu
  \end{pmatrix}
\end{equation}
If $\gamma \geq 1$ and $\mu < \frac{\lambda_1(\Lambda(\mathbf{B})}{2\gamma}$ (or equivalently, $(\Lambda(\mathbf{A}^\intercal), \mathbf{z})$ a BDD$_{1/(2\gamma)}$-instance), then $\Lambda(\mathbf{B}')$ contains a $\gamma$-unique shortest vector $\mathbf{z}' = \left[(\mathbf{A}^\intercal\mathbf{s} - \mathbf{z})^\intercal, -\mu\right]^\intercal = \left[-\mathbf{e}^\intercal, -\mu\right]^\intercal$. 
The statement can be proven by showing by contradiction that all vectors $\mathbf{v} \in \Lambda{\mathbf{B}'}$ that are independent of $\mathbf{z}'$ satisfy $\| \mathbf{v}\| \geq \lambda_1{\Lambda(\mathbf{B})}/\sqrt{2} > \sqrt{2}\gamma \mu = \gamma \|\mathbf{z}'\|$ (see Section 4 of \cite{LM09} for more details). Note that the reduction can be done in polynomial time (Theorem 4.1 in \cite{LM09}).% TODO possibly sketch the proof
Hence, from $\mathbf{z}'$ we can recover the error vector $\mathbf{e}$ and thereby the secret vector $\mathbf{s} = \mathbf{z} - \mathbf{e} \mod q$. 

A solution to $\gamma$-uSVP can be found by reducing it to $\kappa$-HSVP where $\gamma = \kappa^2$ \cite{APS15}. Various algorithms, in particular, lattice reduction algorithms, exist to solve $\kappa$-HSVP. If we are able to solve a linear number of $\kappa$-HSVP instances that correspond to a $\kappa^2$-approximate SVP instance, we can construct a solution the latter (see \cref{def:gammaSVP}, see Section 1.2.21 in \cite{Lov87} for more details). 
Consider any lattice with uSVP structure. In exactly one direction, that is, in the direction of its unique shortest vector, the lattice has vectors that are significantly smaller than in other directions. A lattice reduction algorithm that yields a sufficiently good output basis quality, therefore, must return some small vector in the desired direction. 
Let $\mathbf{v}$ be a solution to SVP$_\kappa^2$, i.e. $\|\mathbf{v}\| \leq \kappa^2 \lambda_1(\Lambda)$. All other vectors $\mathbf{w}\in \Lambda$ that are not multiples of a shortest vector have length $\|\mathbf{w}\| \geq \lambda_2(\Lambda) > \kappa^2\lambda_1(\Lambda)$. Thus, we obtain a solution to $\gamma$-uSVP and, as shown above, we can reconstruct the secret vector to solve LWE. 

% TODO: in practice: GN08



\subsection{Meet-in-the-Middle \cite{APS15}}
% simple description in AFFP14 sec 6



\subsection{Arora-Ge \cite{AG11}}




\section{Algorithms for Solving SIS}
Recall that the SIS$_{n, q, m, \beta}$ problem asks to find a short vector $\mathbf{s} \in \mathbb{Z}_q^m$ of norm $\|\mathbf{s}\| \leq \beta$ such that such that $\mathbf{A} \cdot \mathbf{s} = 0 \; \text{mod } q$ for some uniformly distributed matrix $\mathbf{A}^{n\times m}$. Solving SIS is equivalent to finding a short vector in the dual lattice $\Lambda(\mathbf{A}^\intercal)^{\perp} = \{ \mathbf{y} \in \mathbb{Z}^m \mid \mathbf{A} \mathbf{y} = \mathbf{0} \mod q\}$. 


\subsection{Lattice Reduction \cite{MR09, RS10}}
\subsubsection{MR variant \cite{MR09}} \label{sec:mr-variant}
Our first approach to solving SIS follows quite naturally. Given $\mathbf{A}$, we can efficiently compute the basis $\mathbf{B} (\mathbf{B}^\intercal \mathbf{B})^{-1}$ of the dual lattice $\Lambda(\mathbf{A}^\intercal)^{\perp}$ in polynomial time using Gauss-Jordan elimination or other more modern algorithms.

We can then apply a lattice reduction algorithm and obtain a basis with root Hermite factor $\delta$. The first basis $\mathbf{b}_1$ vector of the reduced basis has length $\|\mathbf{b}_1\| = \delta^m \text{det}(\Lambda(\mathbf{A}^\intercal)^{\perp})^{1/m}$. We can see that $\delta$ depends on the subdimension $m$ which we want to be ideal in order to minimize the cost of the lattice reduction by relaxing $\delta$. 

We further assume that $\text{det}(\Lambda(\mathbf{A}^\intercal)^{\perp}) = \text{Vol}(\Lambda(\mathbf{A}^\intercal)^{\perp}) = q^n$ (see \cite{MR09} for more details). For $q$ prime and $m$ much larger than $n$ we have that the rank of $\mathbf{A}$ is $n$ as the rows of $\mathbf{A}$ are with high probability linearly independent. The nullity or the dimension of the kernel of $\mathbf{A}$ is $m-n$ and as a result the dual lattice has $q^{m-n}$ points in $\mathbb{Z}_q^m$. Consider the fundamental domain $D = \mathcal{P}(\Lambda(\mathbf{A}^\intercal)^{\perp})$ and the fact that $\Lambda(\mathbf{A}^\intercal)^{\perp} + (D \mod q) = \mathbb{R}^m/q\mathbb{R}^m$ is a partition. The volume of $\mathbb{R}^m/q\mathbb{R}^m$ is given by $q^m =|\Lambda(\mathbf{A}^\intercal)^{\perp}||D \mod q|$ and thus
\begin{equation}
  \text{det}(\Lambda(\mathbf{A}^\intercal)^{\perp}) = |D \mod q| = \frac{q^m}{|\Lambda(\mathbf{A}^\intercal)^{\perp}|} = \frac{q^{m}}{q^{m-n}} = q^n.
\end{equation}


We now have our first equation 
\begin{equation}\label{eq:mr-delta}
  \|\mathbf{b}_1\| = \delta^m q^{\frac{n}{m}},
\end{equation}
which becomes minimal for $m = \sqrt{n \log q / \log \delta}$.
\begin{theorem}[Optimal subdimension $m$ \cite{MR09}]
  Given a $q$-ary scaled dual lattice  $\Lambda(\mathbf{A}^\intercal)^{\perp}$ defined by a matrix $\mathbf{A} \in \mathbb{Z}^{n \times m}$ with $m$ sufficiently larger than $n$ and a prime $q$. Then a lattice reduction algorithm yields an optimal output if performed in subdimension 
  \begin{equation}
    m' = \sqrt{\frac{n \log q}{\log \delta}}. \label{eq:mr-m}
  \end{equation} 
\end{theorem}

Higher dimension increase the complexity of the reduction algorithms and lower dimensions may cause a lack of sufficiently short lattice vectors \cite{MR09}. In contexts in which \cref{eq:mr-delta} does not hold, we may still choose $m$ as in \cref{eq:mr-m} heuristically. Removing columns from $\mathbf{A}$ does not greatly impact our results since we can just set the corresponding components of the secret vector $s$ to zero. We reformulate \cref{eq:mr-delta} a bit:
\begin{align}
  \|\mathbf{b}_1\| = \delta^m q^{\frac{n}{m}} \iff&\log \beta = m \log \delta + \frac{n \log q}{m}\\
  \iff&\log \delta = \frac{\log \beta}{m} - \frac{n \log q}{m^2} \label{eq:mr-log-delta}
\end{align}

We continue by plugging \cref{eq:mr-m} into \cref{eq:mr-log-delta}:
\begin{align}
  \log \delta = \frac{\log \beta}{\sqrt{\frac{n \log q}{\log \delta}}} - \frac{n \log q}{\left(\sqrt{\frac{n \log q}{\log \delta}}\right)^2} \iff&\log \delta = \frac{\log \beta}{\sqrt{\frac{n \log q}{\log \delta}}} - \log \delta\\
  \iff&2\log \delta = \frac{\log \beta}{\sqrt{\frac{n \log q}{\log \delta}}}\\
  \iff&\log \delta = \frac{\log^2 \beta}{4n \log q}
\end{align}

\begin{theorem}[Optimal subdimension $m$ \cite{MR09}]
  Given a $q$-ary scaled dual lattice $\Lambda(\mathbf{A}^\intercal)^{\perp}$ defined by a matrix $\mathbf{A} \in \mathbb{Z}^{n \times m}$ with $m$ sufficiently larger than $n$ and a prime $q$. Then a lattice reduction algorithm performed in its optimal subdimension achieves a log root Hermite factor of  
  \begin{equation}
    \log \delta = \frac{\log^2 \beta}{4n \log q}. \label{eq:mr-log-RHF}
  \end{equation} 
\end{theorem}

To estimate the cost of the lattice reduction for SIS, we call a function from the \textit{Estimator} to find the required block size $k$ such that BKZ achieves root Hermite factor $\delta$ and apply a cost model with the optimal subdimension $m'$ and block size $k$. 
% TODO: describe how k is computed?

Note that for LWE we have $\alpha, q$ as input parameters instead of a bound. We can convert $\alpha$ to a required bound $\beta = \frac{1}{\alpha} \sqrt{\ln (\frac{1}{\epsilon})/ \pi}$ such that the success probability of solving an LWE instance is given by $\epsilon$ (Corollary 2 in \cite{APS15}). The \textit{Estimator} uses a rinse and repeat strategy to find the best tradeoff between runtime and success probability.

% TODO: fall unterscheidungen, m' > m oder delta zu klein





\subsubsection{RS variant \cite{RS10}}
% permissive form of distiguishing attack in \cite{MR09}, adversarial advantage is about $2^{-72}$ \cite{LP11}.

A similar approach is described in \cite{RS10}. The optimal subdimension and required root Hermite factor are given by a slightly different expression. Apart from that the attack works as described in \cref{sec:mr-variant}.

\begin{theorem}[Optimal subdimension $m$ (\cite{RS10}, Conjecture 2)]
  For every $n \geq 128,$ constant $c \geq 2, q \geq n^c, m = \Omega(n \log_2(q))$ and $\beta < q$, the best known approach to solve SIS with parameters ($n, m, q, \beta$) involves solving $\delta$-HSVP in dimension $m' = \min(x : q^{2n/x} \leq \beta)$ with $\delta = \sqrt{d}{\beta / q^{n/m'}}$.
\end{theorem}

We reformulate the expression for $m'$
\begin{align}
  q^{2n / m'} &\leq \beta \\
  \frac{2n}{m' \log(q)} &\leq \beta \\
  m' &\geq \frac{2n \log(q)}{\log(\beta)}
\end{align}
and obtain $m' = \left\lceil \frac{2n \log(q)}{\log(\beta)} \right\rceil$. If $m' > m$, we take $m' = m$. 

The root Hermite factor $\delta$ must be larger than $1$ for the reduction to be tractable. From $\delta = \sqrt{d}{\beta / q^{n/m'}} \geq 1$ it follows that we need $m' \geq n \log_2(q) / \log_2(\beta)$. 

% TODO: add some comments
% TODO: give intuition about the conjecture???


\subsection{Combinatorial Attack \cite{MR09}}
Micciancio and Regev also describe a combinatorial method for solving SIS \cite{MR09}. 

- matrix $\mathbf{A} \in \mathbb{Z}_q^{n \times m}$, dual lattice $\Lambda(\mathbf{A}^\intercal)^{\perp}$, find lattice vector bounded by $b$ in all $m$ coefficients

- $2^k$ sets of column vectors for some $k$, $m/2^k$ columns in each set
- compute all linear combinations $c_1 \mathbf{b}_1 + \cdots + c_{m/2^k}\mathbf{b}_{m/2^k}$ for column vectors $\mathbf{b}_i$ in each set such that the coefficients are bounded by $b$, i.e. $|c_1| \leq b$ to produce $2^k$ new sets of $l=(2b+1)^{m/2^k}$ vectors % TODO: at most or exactly?
- combine sets pairwisely: for each vector in first set $\mathbf{x}$ combine with each vector $\mathbf{y}$ in second set, if first $\log_q l$ components in $\mathbf{x} \pm \mathbf{y}$ are zero put in combined set
- combined sets have size $l$ as $\log_q l$ components $q^{\log_q l}$ values % TODO rewrite
- now $2^{k-1}$ sets, repeat $k-1$ times => vectors in last result set have zero entries in first $k \log_q l$ components % TODO

Choose $k$ such that 
\begin{equation}
  n \approx (k+1) \log_q l = (k+1) \log_q (2b+1)^{m/2^k} \iff \frac{2^k}{k+1} \approx \frac{m \log(2\beta + 1)}{n \log(q)}
\end{equation}
- then approximately $\log_q l$ coodinates in result set that are not cancelled out => we should obtain at least one zero vector as at there are most $l$ different vectors
- zero vector is linear combination with entries bounded by $b$


\begin{align}
  \frac{2^k}{k+1} &\approx \frac{m \log(2\beta + 1)}{n \log(q)}\\
  \text{diff} &= \text{abs}\left(\frac{2^k}{k+1} - \frac{m \log(2\beta + 1)}{n \log(q)}\right)
\end{align}

To find an optimal $k$, we iterate over $k$ starting from $k=1$ and calculate the difference $\text{diff}$. When $\text{diff}$ does not decrease for 10 iteration steps, we stop and take the current $k$.

We make a conservative estimate of the cost by estimating the number of operations needed to create the initial lists as the overall cost is dominated by this parameter. Each of the $2^k$ lists contains $L$ vectors. The cost for any operation on a list element is at least $\log_2(q) \cdot n$. Hence, the total cost is $2^k * L * \log_2(q) * n$.



% TODO maybe change definition of lattice so that we can write A instead of A^\intercal.


\section{Tool}
class for distributions... from section this modelling, problems, generic search... Überblick, wie verwendbar,
automatische norm umwandlung,
sonstige features

\subsection{Runtime and Cost Comparison}
defaults...
schnellste, beste => effizient, etc. parallel... problem reductions...



\chapter{Usage Examples}
\section{Two Problem Search}
% TODO: based on \cite{BDLOP18}
two problems lwe sis, what they are, how to solve it by the tool
\section{TODO: find other schemes to apply}


\chapter{Conclusion}

\printbibliography

All links were last followed on October 1, 2021.

%\appendix

\pagestyle{empty}
\renewcommand*{\chapterpagestyle}{empty}
\Versicherung
\end{document}